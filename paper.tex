\documentclass[a4paper]{article}

\usepackage[utf8]{inputenc}
\usepackage[T1]{fontenc}
\usepackage{textcomp}
%\usepackage[dutch]{babel}
\usepackage{amsmath, amssymb, amsthm}
\usepackage{geometry}
\usepackage{tikz-cd}
\usepackage{mdframed}
\usepackage{color}
\usepackage{microtype}
\usepackage{hyperref}
\usepackage{cleveref}
\usepackage{todonotes}
\usepackage[backend=biber]{biblatex}
\addbibresource{references.bib}
% figure support
\usepackage{import}
\usepackage{xifthen}
\usepackage{pdfpages}
\usepackage{transparent}
\newcommand{\incfig}[1]{%
	\def\svgwidth{\columnwidth}
	\import{./figures/}{#1.pdf_tex}
}

\pdfsuppresswarningpagegroup=1


\newtheoremstyle{theoremdd}% name of the style to be used
  {7pt}% measure of space to leave above the theorem. E.g.: 3pt
  {7pt}% measure of space to leave below the theorem. E.g.: 3pt
  {\itshape}% name of font to use in the body of the theorem
  {0pt}% measure of space to indent
  {\bfseries}% name of head font
  {}% punctuation between head and body
  { }% space after theorem head; " " = normal interword spae
  {\thmname{#1}\thmnumber{ #2}\thmnote{ (#3) } \hspace{.1cm}}

\newtheoremstyle{definitiondd}% name of the style to be used
  {7pt}% measure of space to leave above the theorem. E.g.: 3pt
  {7pt}% measure of space to leave below the theorem. E.g.: 3pt
  {\rmfamily}% name of font to use in the body of the theorem
  {0pt}% measure of space to indenlt
  {\bfseries}% name of head font
  {}% punctuation between head and body
  { }% space after theorem head; " " = normal interword space
  {\thmname{#1}\thmnumber{ #2}\thmnote{ (#3) } \hspace{.1cm}}

  \newtheoremstyle{remarkdd}% name of the style to be used
  {7pt}% measure of space to leave above the theorem. E.g.: 3pt
  {7pt}% measure of space to leave below the theorem. E.g.: 3pt
  {\rmfamily}% name of font to use in the body of the theorem
  {0pt}% measure of space to indent
  {\itshape}% name of head font
  {}% punctuation between head and body
  { }% space after theorem head; " " = normal interword space
  {\thmname{#1}\thmnumber{ #2} \thmnote{ (#3)} \hspace{.1cm}}


\theoremstyle{theoremdd}
\newtheorem{theorem}{Theorem}[section]
\newtheorem{proposition}[theorem]{Proposition}
\newtheorem{lemma}[theorem]{Lemma}
\newtheorem{conjecture}[theorem]{Conjecture}
\newtheorem{corollary}[theorem]{Corollary}
\newtheorem{property}[theorem]{Property}
\theoremstyle{definitiondd}
\newtheorem{definition}[theorem]{Definition}
\theoremstyle{remarkdd}
\newtheorem{example}[theorem]{Example}
\newtheorem{remark}[theorem]{Remark}
\newtheorem{exercise}[theorem]{Exercise}

\newcommand{\N}{\mathbb{N}}
\newcommand{\Z}{\mathbb{Z}}
\newcommand{\Q}{\mathbb{Q}}
\newcommand{\C}{\mathbb{C}}
\newcommand{\R}{\mathbb{R}}
\newcommand{\ltr}{\par \noindent \framebox[1\width]{ $\implies$ } \hspace{.2cm}}
\newcommand{\rtl}{\par \noindent \framebox[1\width]{ $\impliedby$ } \hspace{.2cm} }
\newcommand{\bigset}[2]{ \left\{ #1 \;\middle|\; #2 \right\} }


\DeclareMathOperator{\LC}{LC}
\DeclareMathOperator{\FC}{FC}
\DeclareMathOperator{\coef}{Coef}
\DeclareMathOperator{\coker}{coker}
\DeclareMathOperator{\Id}{Id}
\DeclareMathOperator{\Ann}{Ann}
\DeclareMathOperator{\im}{Im}
\DeclareMathOperator{\SL}{SL}


\author{Micha\"el Maex}
\date{\today}
\title{Modular Curves}
\begin{document}

\maketitle
\pagebreak

\tableofcontents

\pagebreak

\section{Introduction}

This is the report of the journey of a student learning about elliptic curves. 
This report should be thought of as my notes while working my way through the first chapters of \emph{A First Course in Modular Forms} by Fred Diamond and Jerry Shurman\cite{diamondFirstCourseModular2005a}. 
This book is my main reference. 
Most of this report is paraphrasing this book, but I've added some details where I thought the book was lacking. 
I've also added some visual intuitions I developed to aid my understanding of this material. I hope potential readers will find these useful as well. 

Unfortunately I did not get far enough into the subject to start studying more advanced subjects, due to the large amount of concepts, and relations between these concepts, that need to be understood first (the modular group, congruence subgroups, modular forms, complex tori, elliptic curves, modular curves, moduli spaces, \ldots). 
However this report should cover most if not all the prerequisites for understanding the more advanced topics.

Modular forms, elliptic curves and modular forms are three seemingly unrelated concepts that have surprising connections. 
The goal of this report is to understand these objects and how they are related. 

\section{Modular Forms}
Modular forms are function on the complex half plane $\mathcal{H} $ that are invariant in some sense under the action of a group (or subgroups of) called the modular group.

\subsection{The Modular Group and its Action}
We begin by introducing the modular group and its action on $\mathcal{H} \cup \{\infty\} $.  

\begin{definition}
	The \emph{Modular group} are the $2\times 2$ matrices with integer coefficients and determinant $1$.
	 \[
		 \SL_2(\Z) =   \bigset{\begin{pmatrix} a& b\\ c& d \end{pmatrix}}{a, b, c, d \in \Z, ac - bd = 1}
	.\] 
\end{definition}
It can be shown that this is indeed a group and that it is generated by \[
	\begin{pmatrix}  1 & 1 \\ 0 & 1 \end{pmatrix} 
	\text{ and }
	\begin{pmatrix} 0 & -1 \\ 1 & 0 \end{pmatrix} 
.\] 
For a proof of this we refer to the appendices. 
We now define how this group acts on the complex half plane. 
\begin{definition}
	Let  $\gamma = \begin{pmatrix} a & b \\ c & d \end{pmatrix} $ be any element of $\SL_2\Z$ and $\tau$ be any element of $\mathcal{H} \cup \{\infty\} $.
	The multiplication of these two elements is defined as 
	\[
		\gamma(\tau) = \gamma \cdot \tau = \frac{a \tau + b}{ c \tau + d}
	.\] 
\end{definition}
Note that this group action is well defined. First we have to check that this every $\tau \in \mathcal{H} \cup \{\infty\}  $ gets mapped to another element of $\tau \in \mathcal{H} \cup \{ \infty\}$. 
Note that is is sufficient to check this for the action of the generators, which induce the maps \[
\tau \mapsto  \tau+1 \text{ and } \tau \mapsto \frac{-1}{\tau}
\]
for which it is obvious.
Further, we easily see that $I_2 \cdot \tau = \tau$. 
It also holds that for $\gamma = \begin{pmatrix} a & b \\ c & d \end{pmatrix} $ and $\beta = \begin{pmatrix} f & g \\ h & i \end{pmatrix} $: 
\begin{align*}
	\gamma \left( \beta \cdot  \tau\right) &=  \gamma\cdot \frac{f\tau + g}{h \tau + i} \\
					       &= \frac{a\left( \frac{f \tau + g}{h\tau + i} \right) + b}{c \left( \frac{f \tau + g}{h \tau + i} \right) + d} \\	
					       &= \frac{(af + bh)\tau + ga + bi}{(cf + dh) \tau + cg + di} \\
					       &= \begin{pmatrix} af + bh & ag + bi \\ cf + dh & cg + di \end{pmatrix} \cdot \tau \\
					       &= (\gamma \beta)\cdot \tau 
.\end{align*}

\subsection{Subgroups of the Modular Group} \label{sec:subgroups_of_the_modular_group}
Quite often it will be useful to restrict our focus to subgroups of the modular group. 
The type of subgroups of greatest interest are the so-called \emph{congruence subgroups}.
To define these we first have to define the so called \emph{principal congruence subgroups}.  
\begin{definition}
	For any $N \in \N$ we define the \emph{principal congruence subgroup of level $N$} to be \[
		\Gamma(N) = \bigset{\begin{pmatrix} a & b \\ c & d \end{pmatrix} \in \SL_2(\Z)}{\begin{pmatrix} a & b \\ c & d \end{pmatrix}  \equiv \begin{pmatrix} 1 & 0 \\ 0 & 1 \end{pmatrix}  (\mathrm{mod}\; N)}
	.\] 
\end{definition}

Further we define two subgroups
\begin{definition}
	For every $N \in \N$ we define $\Gamma_0$ and $\Gamma_1$ to be 
	\begin{align*}
		\Gamma_0(N) &= \bigset{\begin{pmatrix} a & b \\ c & d \end{pmatrix} \in \SL_2(\Z)}{\begin{pmatrix} a & b \\ c & d \end{pmatrix}  \equiv \begin{pmatrix} * & * \\ 0 & * \end{pmatrix}  (\mathrm{mod}\; N)} \\
		\Gamma_1(N) &= \bigset{\begin{pmatrix} a & b \\ c & d \end{pmatrix} \in \SL_2(\Z)}{\begin{pmatrix} a & b \\ c & d \end{pmatrix}  \equiv \begin{pmatrix} 1 & * \\ 0 & 1 \end{pmatrix}  (\mathrm{mod}\; N)}
	.\end{align*}
\end{definition} 
Note that 
\[
	\Gamma(N) \subset \Gamma_1(N) \subset \Gamma_0(N) \subset  \SL_2\Z
.\] 
We state without proof that $[\Gamma_1(N): \Gamma(N)] = N$, $[\Gamma_0 (N): \Gamma_1(N)]= \phi(N)$, $[\SL_2\Z: \Gamma_0(N)] = N \prod_{p \mathbin | \N}(1 + 1 /p)$. 
From this it follows that $[\Gamma(N): \SL_2\Z]= N^3 \prod_{p \mathbin | N} 1 - 1 /p^2$.
We will now give the general definition of \emph{congruence subgroups}. 
 \begin{definition}
	 A subgroup $\Gamma \subset \SL_2\Z$ is a \emph{congruence subgroup} if $\Gamma(N) \subset \Gamma$ for some $N \in \N$. In this case we say that $\Gamma$ is a congruence subgroup of level $N$. 
\end{definition}
\subsection{Defining Modular Forms}\label{sec:defining_modular_forms}

Some meromorphic functions $f:\mathcal{H}  \to \hat{\C}$ are in some sense invariant under the action of the modular group (or one of its subgroups). 
These functions are the \emph{modular forms}.

\begin{definition}
	Let $k$ be an integer. 
	A meromorphic function $f: \mathcal{H}  \to \C$ is \emph{weakly modular of weight $k$ } if \begin{equation} \label{eq:weakly_modular}
		f(\gamma(\tau)) = (ct + d)^{k}f(\tau)
	,\end{equation}
	for all $\gamma = \begin{pmatrix} a & b \\ c & d \end{pmatrix}  \in \SL_2(\Z)$ and $\tau \in \mathcal{H} $. 
\end{definition}

It's clearly enough so check \eqref{eq:weakly_modular} only for the generators of $\SL_2\Z$. I.e. check that
\begin{equation}\label{eq:sufficient_weakly_modular}
	f(\tau + 1) = f(\tau) \text{ and } f\left(-\frac{1}{\tau}\right)  = \tau^{k}f(\tau)
.\end{equation} 
From this we see that weakly modular functions are $\Z$ periodic. 

We will now introduce the weight-$k$ operator on functions $[\gamma]_k$.
 \begin{definition}
	 Let $f: \mathcal{H}  \to \C$ be a function,  $\gamma = \begin{pmatrix} a & b \\ c & d \end{pmatrix} \in \SL_2\Z$ and $k \in \Z$ then we define the function $f[\gamma]_k$ to be 
	  \begin{align*}
		  f[\gamma]_k: \mathcal{H}  &\longrightarrow \C \\
		  \tau &\longmapsto (c\tau + d)^{-k}f(\gamma(\tau))
	 .\end{align*}
\end{definition}
This allows us to characterize  \emph{weakly modular of weight  $k$} as follows.
 \begin{lemma}
	 A function $f: \mathcal{H}  \to \C$ is weakly modular of weight $k$ if and only if for every $\gamma \in \SL_2$: \[
		 f[\gamma]_k = f
	 .\] 
\end{lemma}
In light of this characterization we extend to definition of weakly modular to congruence subgroups of $\SL_2\Z$. 

\begin{definition}
	Let $\Gamma$ be a congruence subgroup of $\SL_2\Z$. We say that a function $f: \mathcal{H}  \to \C$ is \emph{weakly modular of weight $k$ with respect to $\Gamma$} if for every  $ \gamma \in \Gamma$:
	 \[
		 f[\gamma]_k = f 
	.\] 
\end{definition}

We are now ready to give the definition of a modular form. 
\begin{definition}\label{def:modular_form}
	Let $\Gamma$ be congruence subgroup of $\SL_2\Z$.
	A function $f: \mathcal{H}  \to \C$ is a \emph{modular form of weight $ k$ with respect to $\Gamma$} if 
	\begin{enumerate}
		\item $f$ is holomorphic on $\mathcal{H} $,
		\item $f$ is weakly modular of weight $k$ with respect to $\Gamma$, 
		\item $f[\gamma]_k$ is holomorphic at $\infty$ for every $\gamma \in \SL_2\Z$. 
	\end{enumerate}
	We denote the set of modular froms of weight $k$ with respect to $\Gamma$ as $\mathcal{M} _k(\SL_2\Z)$. 
	
	If $\Gamma = \SL_2\Z$ we say simply that $f$ is a modular form of weight $k$. 
\end{definition}
The stament `holomorhic at $\infty$' deserves some explanation. 
One can show that any congruence subgroup contains a matrix of the form $\begin{pmatrix} 1 & k \\ 0 & 1 \end{pmatrix} $ ($\Gamma(N)$ contains  $\begin{pmatrix} 1 & n \\ 0 & 1 \end{pmatrix} $). 
If we choose this $k$ to be minimal we see that $f(\tau) = f(\tau + k)$. 
So $f$ is $k\Z$-periodic. It would be desirable if $f$ allowed a Fourrier expansion.  
There are many equivalent ways of defining `holomorphic at $\infty$', but for our purposes we will take this to mean that $f$ allows a Fourier expansion \footnote{The real picture behing holomopich at $\infty$ is that, due to $f$ being $k\Z$-periodic, $H$ can be 'rolled up' on the unit disk, where $\infty$ gets mapped to the center. In this picture  $f$ being holomorphic at $\infty$ means that  $f$ allows an extension which holomorphic on the center of the disk.}, i.e.\ \[
	f(\tau) = \sum_{n=0}^{\infty} a_n e^{n 2\pi i \tau / k}
,\]
for some $a_n \in \C$.
Another equivalent statement is that $\lim_{\im(\tau) \to \infty} f(\tau)$ exists and is finite.

Note that it is only necessary to verify the 3rd point of \cref{def:modular_form} for a representative of every coset $\alpha \Gamma$ as by the weak modularity  $f[\alpha]_k = f[\alpha \gamma]_k$ for every  $\gamma \in \Gamma$. 
If  $f$ is a modular form (so $\Gamma = \SL_2\Z$) then we just have to verify that $f$ is holomorphic at $\infty$.
Note that $\mathcal{M}(\Gamma)$ is a $\C$-vector space. 
It is the case that the last condition of \cref{def:modular_form} is what makes $\mathcal{M} (\SL_2\Z)$ of finite dimension, but I will not show this. 
\begin{lemma}\label{lem:product_modular_forms}
	Let $f$ be a modular form of weight $k$ and $g$ a modular form of weight $l$, then 
	$fg$ is a modular form of weight  $k + g$.
\end{lemma}
\begin{proof}
	The three conditions from definition \ref{def:modular_form} must be verified. 
	\begin{enumerate}
		\item The product of holomorphic functions is holomorphic. So $ fg$ is holomorphic on $\mathcal{H} $. 
		\item Take any  $\gamma = \begin{pmatrix} a & b \\ c & d \end{pmatrix} \in \SL_2\Z$ and $\tau \in \mathcal{H} $. Then 
			\begin{align*}
				(fg)(\gamma \cdot \tau) &= f(\gamma \cdot \tau) g(\gamma \cdot \tau)   \\
							&= (c\tau + d)^{k} f(\tau) (c \tau)^{l} g(\tau) \\
							&= (c\tau + d)^{k+l} (fg)(\tau) \\
			.\end{align*}
			So $fg$ is weakly modular of weight $k+l$. 
		\item We know that $\lim_{\im(\tau) \to \infty} f(\tau)$, $\lim_{\im(\tau) \to \infty} g(\tau)$ exists and is finite. Hence $\lim_{\im(\tau) \to \infty} fg (\tau)$ exists and is finite.		
	\end{enumerate}
\end{proof}

Finally we introduce a special type of modular called a cusp. The names comes from connection to modular curves that we will discuss later.
\begin{definition}\label{def:cusp}
	A \emph{cusp form} is a modular form $f$ (with respect to a congruence subgroup  $\Gamma$) such that for every  $\alpha \in \SL_2\Z$  the leading coefficient in its Fourier expansion is 0. 
	 The $\C$-vector space of cusp forms of weight $k$ is denoted $\mathcal{S} _k(\Gamma)$.
\end{definition}
Again, it is sufficient to check this for a representative for every coset $\alpha \Gamma$.

\subsection{Examples of Modular Forms}
We may wonder whether these types of functions exist. One trivial modular form of weight $k$ is the zero function, but the question remain whether there are non trivial modular forms of a given weight.
If $k$ is odd the answer is negative as $f(\tau) = f(-I_2 \tau) = (-1)^{k} f(\tau) = 0$. 
If we know a few modular forms of small weight, we can construct modular forms of higher weights using \cref{lem:product_modular_forms}.

However, we can construct a non trivial modular form for every even weight greater than 2.
\begin{definition}
	The \emph{Eisenstein series of weight $k$} ($k$ even and  $k >2$) is the function
	\begin{align*}
		G_k: \mathcal{H} &\longrightarrow \C \\
		\tau &\longmapsto \sum_{\substack{(c,d) \in \Z^2 \\ (c,d)\ne (0,0) }} \frac{1}{(c\tau + d)^{k}}
	.\end{align*}
\end{definition}
We of course have to check that this is indeed a modular form of weight $k$. 
Basic analysis can be used to show that the summation is absolutely convergent for every  $\tau \in \mathcal{H}$ and uniformly convergent on compact sets.
It follows that $G_k$ is holomorphic.
Recall from \cref{sec:defining_modular_forms} that it is sufficient to check weak modularity for  $\gamma = \begin{pmatrix} 1 & 1 \\ 0 & 1 \end{pmatrix} $ and $\gamma = \begin{pmatrix} 0 & -1 \\ 1 & 0  \end{pmatrix} $, i.e.\ we have to show that \cref{eq:sufficient_weakly_modular} holds.
We see 
\begin{align*}
	G_k(\tau + 1) &= \sum_{\substack{(c, d) \in \Z^2 \\ (c, d) \ne (0,0)}} \frac{1}{(c (\tau + 1) + d)^{k}}\\
			 &= \sum_{\substack{(c, d) \in \Z^2 \\ (c, d) \ne (0,0)}} \frac{1}{(c \tau + (d+c))^{k}}\ \\
.\end{align*}
But this is just a permutation of the terms of the previous sequence, as $\Z^2 \to \Z^2: (c,d) \mapsto (c, c+ d)$ is a isomorphism of groups.
Hence $G_k(\tau + 1) = G_k(\tau)$.
We have to check the second equation as well.
\begin{align*}
	G_k\left(\frac{-1}{\tau}\right) &=  \sum_{\substack{(c,d) \in \Z^2 \\ (c,d) \ne 0}} \frac{1}{(-c / \tau + d)^{k}}\\
			     &= \tau^{k} \sum_{\substack{(c, d) \in \Z^2 \\ (c,d)}} \frac{1}{(d \tau - c)^{k}} \\
			     &= \tau^{k}G_k(\tau) \\
\end{align*}
The last equality is again true as we can permute the terms as the summation is absolutely convergent.

A lengthy calculation that we will omit shows that \begin{equation}\label{eq:expansion_eisenstein}
	G_k(\tau) = 2 \zeta(k) + 2 \frac{(2\pi i)^{k}}{(k-1)!}\sum_{n = 1}^{\infty} \sigma_{k-1}(n)  e^{n 2 \pi i \tau}
,\end{equation}
whenever $ k>2$ and $k$ is even, where $\zeta$ is the Riemann zeta function and $\sigma$ is the sum of divisors function, defined as
\[
	\sigma_{k}(n) = \sum_{\substack{d \mathbin | n \\ d > 0}} d^{k}
.\] 
We've yet to find an example of a cusp. We can do this by taking the right linear combination of two independent elements of $\mathcal{M} _k(\SL_2\Z)$ in a way such that the first coefficient cancels.
In this spirit we define the \emph{discriminant function}
\begin{align*}
	\Delta = (60 G_4)^3 - 27(140G_6)^2
.\end{align*}
Note that this a modular form of weight 12. 
Using \cref{eq:expansion_eisenstein} we see that the first two coefficients are $0, (2 \pi)^{12}$. 
Hence it is a non-trivial cusp of weight 12. 

\section{Lattices and Complex Elliptic Curves}
The main idea of this section is to construct tori by considering $\C$ as an additive group modulo a certain kind of embedding of $\Z^2$. This embedding of $\Z^2$ is called a lattice.  
\begin{definition}
	A \emph{lattice in $\C$} is additive subgroup $\Lambda = \omega_1 \Z \oplus \omega_2 \Z$ where $\omega_1, \omega_2 \in \C$ are linearly independent over $\R$ and $\omega_1 / \omega_2  \in \mathcal{H}$.
\end{definition}
\begin{definition}
	A \emph{complex torus} is a quotient of the complex plane by a lattice, \[
		\C / \Lambda = \bigset{z + \Lambda}{z \in \C} 
	.\] 
\end{definition}
This object is called as torus because of a geometric intuition. 
If one considers the parallelogram spanned by $\vec{\omega_1}, \vec{\omega_2}$ we see that the opposing edges are `glued' together. 
So topologically this space is homeomorphic to the torus (see \cref{fig:complex_torus}). 

\begin{figure}[h]
	\centering
	\incfig{lattice_complex_plane}
	\caption{A lattice generated by $\omega_1$ and $\omega_2$. The grey part shows a fundamental parallelogram of the lattice which represents the complex torus $\C / \Lambda$. The arrows on the edges shows how the edges are `glued' together to form a torus.} \label{fig:complex_torus}
\end{figure}

The modular group tells us when two two lattices are equal. 
\begin{lemma}\label{lem:equality_of_lattices}
	Let $\Lambda = \omega_1\Z \oplus \omega_2\Z$ and $\Lambda' = \omega_1'\Z \oplus \omega_2\Z$ be two lattices. 
	Then $\Lambda = \Lambda'$ if and only if there exists an $\begin{pmatrix} a & b \\ c& d \end{pmatrix} \in \SL_2\Z$ such that \[
	\begin{pmatrix} \omega_1' \\ \omega_2' \end{pmatrix}  = 
	\begin{pmatrix} a & b \\ c& d \end{pmatrix} 
	\begin{pmatrix} \omega_1 \\ \omega_2 \end{pmatrix} 
	.\] 
\end{lemma}
\begin{proof}[Proof. (exercise 1.3.1)]
	\ltr
	We know that $\omega_1' \in \Lambda$, there there are $a, b \in \Z$ such that $\omega_1' = a \omega_1 + b \omega_2$. Similarly there are $c, d \in \Z$ such that $\omega_2' = c \omega_1 + d \omega_2$. 
	So  \[
	\begin{pmatrix} \omega_1' \\ \omega_2' \end{pmatrix}  = 
	\begin{pmatrix} a & b \\ c& d \end{pmatrix} 
	\begin{pmatrix} \omega_1 \\ \omega_2 \end{pmatrix} 
	.\] 
	We still have to argue that the matrix is in $\SL_2\Z$, i.e. has determinant 1. 
	Analogously we can find $f, g, h, i \in \Z$ such that \[
	\begin{pmatrix} \omega_1  \\ \omega_2 \end{pmatrix}  = 
	\begin{pmatrix} f & g \\ h& i \end{pmatrix} 
	\begin{pmatrix} \omega_1' \\ \omega_2' \end{pmatrix} 
	.\]
	Hence \[
	\begin{pmatrix} \omega_1 \\ \omega_2 \end{pmatrix}  = 
	\begin{pmatrix} f & g \\ h & i \end{pmatrix} 
	\begin{pmatrix} a & b \\ c & d  \end{pmatrix} 
	\begin{pmatrix} \omega_1 \\ \omega_2 \end{pmatrix} 
	.\]  
	From this it follows that the determinants must be either $1$ or $-1$. 
	We can see that is must be $1$ as the imaginary part of $\omega_1 / \omega_2$ and $\omega_1' / \omega_2'$ must be both positive.
	\rtl
	We can use the fact that the lattices are groups. We know that the $\Lambda'$ is generated by $\omega_1'$ and $\omega_2'$.
	We further know that $\omega_1' = a \omega_1 + b \omega_2$ so $\omega_1' \in \Lambda$. 
	Similarly we see that $\omega_2' \in \Lambda$.
	So $\Lambda' \subset  \Lambda$.

	By inverting the matrix a completely analogous argument shows that $\omega_1 \in \Lambda'$ and $\omega_2 \in \Lambda'$. So $\Lambda \subset \Lambda'$.
\end{proof}

A complex torus is a Riemann surface. This a not a report about Riemann surfaces, but to not completely shove everything under the rug I will quickly explain what a Riemann Surface is.
Recall that that smooth manifolds are spaces that are locally homeomorphic to $\R^{n}$ and equipped with a smoothly compatible atlas. 
In the same spirit Riemann surfaces are objects that look locally like $\C$ and equipped with an atlas that is holomorphicly compatible.
The complex tori are Riemann surfaces. 

I will mention one theorem about Riemann surfaces without proof. 
\begin{theorem}
	Let $f$ be a holomorphic map between compact Riemann surfaces. Then $f$ is either a map to a single point or a surjection.
\end{theorem}

\subsection{Morphisms between Complex Tori}

In our study of complex tori, holomorphic maps between them are of great importance. 
In light of this we want to know how such a maps look like. It turns out they are linear transformations with some restrictions. 
\begin{proposition}
	Let $\phi: \C \ \Lambda \to \C \ \Lambda'$ be a holomorphic map between complex tori. 
	Then $\phi(z + \Lambda) = mz + b + \Lambda'$ for some $m, b \in \C$ such that $m \Lambda \subset  \Lambda'$. 
	The map is invertible if and only if $m \Lambda = \Lambda'$.
\end{proposition}
\begin{proof}
	The idea is to use topology to lift the map. As $\R^2 \simeq \C$ is the universal covering space of the torus we know that we can lift $\phi$ to a holomorphic map $\tilde \phi: \C / \Lambda \to \C$. We then can make this into a homolomorphism by first projecting $\C$ onto $\C / \Lambda$. We get a holomorphic map  $\tilde \phi: \C \to \C$.
	Take any $\lambda \in \Lambda$. 
	Then the map $f_\lambda(z) = \tilde\phi(z + \lambda) - \tilde \phi(z)$. We know that $f_{\lambda}$ is continous, but we also know that is must map to an element of $\Lambda'$, which is discrete, so it must be constant. 
	We can differentiate to find that $\tilde\phi'(z+ \lambda) = \tilde\phi'(z)$ for every $\lambda \in \Lambda$, hence it is  $\Lambda$ periodic. 

	This makes $\tilde\phi'$ bounded so its constant. 
	So $\tilde \phi(z) = mz + b$ for some $m, b \in \C$. 
	Recall that $f_{\lambda}$ is is constant and equal to an element of $\Lambda'$. 
	$\phi(z+\lambda) - \phi(z) = m\lambda \in \Lambda'$. 
	Hence $m \Lambda \subset \Lambda'$.

	We will now prove the second part of the theorem.
	\ltr Suppose that the inclusion  $m\Lambda \subset \Lambda'$ is proper. 
	Take a $\lambda \in \Lambda' \setminus m\Lambda$. 
	So $\lambda / m \not\in  \Lambda$. But $\phi(\lambda m) = b + \Lambda' = \phi(\lambda)$. Thus $\phi$ is not injective. So there is no inverse.
	\rtl It can be easily checked that the map $\psi: \C / \Lambda' \to \Lambda: z + \Lambda \mapsto  \frac{z-b}{m} + \Lambda$ is an inverse. 
\end{proof}

\subsection{Modular Forms and Complex Tori} \label{sec:modular_forms_and_complex_tori}
We now have a notion of \emph{morphisms} between complex tori. We can now ask ourselves the question how equivalence classes of isomorphic complex tori look like.
It turns out that every complex torus $\C / \Lambda$ is \emph{isomorphic} to a torus of the form $\C / \Lambda_\tau$, where $\Lambda_\tau = \tau \Z \oplus \Z$, for some $\tau \in  \mathcal{H}$. 
Indeed, by definition $\Lambda = \omega_1 \Z \oplus \omega_2 \Z$ for some $\omega_1, \omega_2 \in \C$ such that $\omega_1 / \omega_2 \in \mathcal{H} $. Let $\tau = \omega_1 / \omega_2$. 
Then $\frac{1}{\omega_1} \Lambda = \Lambda_\tau$. 
So the map  \[
\phi: \frac{\C}{ \Lambda} \to \frac{\C}{\Lambda_\tau}: z + \Lambda \mapsto \frac{z}{\omega_1} + \Lambda_\tau
\]
is an isomorphism.

This $\tau$ however may not be unique. Suppose that for some $\tau' \in \mathcal{H} $ the torus $\C / \Lambda_{\tau'}$ is also isomorphic to $\C / \Lambda$. 
In that case $\C / \Lambda_{\tau'}$ is also isomorphic to $\C / \Lambda_{\tau}$. 
Then for some $m$ it must be the case that $m \Lambda_{\tau'} = m\tau' \Z \oplus m \Z = \Lambda_\tau$. 
By \cref{lem:equality_of_lattices} this implies that there is some $\gamma = \begin{pmatrix} a & b \\ c& d \end{pmatrix}  \in \SL_2\Z$ \[
\begin{pmatrix} m\tau' \\ m \end{pmatrix}  = \begin{pmatrix} a & b \\ c& d \end{pmatrix}  \begin{pmatrix} \tau \\ 1 \end{pmatrix} 
.\] 
So \[
\tau' = \frac{m \tau'}{m} = \frac{a \tau + b}{c \tau + d} = \gamma\cdot \tau
.\] 
Hence we see that $\tau$ and $\tau'$ lie in the same orbit of the action of $\SL_2\Z$. 
Converesely $\C / \Lambda_\tau$ and $\C / \Lambda_{\gamma\cdot \tau}$ will always be isomorhic.
Hence we conclude that there is a bijectice correspondence of isomorphism classes of complex tori and orbits of $\SL_2\Z$ acting on $\mathcal{H} $.\[
\begin{tikzcd}
	\left\{ \substack{\text{isormorphism classes} \\ \text{of complex tori}} \right\} \arrow[r, leftrightarrow] & 
	\mathcal{H} / \SL_2\Z\ 
\end{tikzcd}
.\] 
\subsection{Isogenies}
If we only consider the maps where the contsant term $b$ is $0$ then these maps are clearly group morphisms between $\C / \Lambda$ and  $\C / \Lambda'$. 
As a result, the a holomorphic group morphism between these two tori exists if and only if $m\Lambda = \Lambda'$ for some $m \in \C$. 
A nonzero holomorphic group morphism between complex tori is called an \emph{isogeny}.

\begin{lemma}
	Isogenies are surjective and have a finite kernel. 
\end{lemma}
\begin{proof}
	Let $\phi: \C / \Lambda \to \C / \Lambda'$ be an isogeny. 
	From the previous discussion we know that $\phi(z+\Lambda) = mz + \Lambda'$ for some $m \in\C$ such that $m \Lambda \subset  \Lambda'$. 
	For every  $x + \lambda'$ it's clear that  $\phi( x / m + \Lambda) = x + \Lambda'$. 
	So  $\phi$ is surjective. 
	Suppose that $\phi(z + \Lambda) = 0 + \Lambda'$. Then  $m z \in \Lambda'$. So  $z \in  \Lambda'/m$. 
	We also know that  $\Lambda \subset \Lambda' / m$. 
	The inclusion map is an injective group morphism from $\Z^2$ to  $\Z^2$ which must be of finite index. 
	So there are only finitely many $z + \Lambda$ such that  $\phi(z + \Lambda) = 0$.
\end{proof}
To gain a more intuitive understanding of isogenies, it is helpful to think of them as functions on the plane that map lattices \emph{into} lattices.
In this context, we see that an isogeneny can scale and rotate the lattice but we cannot skew it as illustrated in \cref{fig:isogeny}.
\begin{figure}[h]
	\centering
\incfig{isogeny_complex_plane}
\caption{An isogeny from a lattice $\Lambda$ generated by $\omega_1$ and $\omega_2$ to a lattice $\Lambda'$ generated by $\omega_1'$ and $\omega_2'$}.
\label{fig:isogeny}
\end{figure}

\begin{example}
	The so called \emph{multiply by integer maps} are isogenies. These are maps of the form \[
		[N]: \C / \Lambda \to \C / \Lambda: z + \Lambda \mapsto  Nz + \Lambda
	,\] where $N$ is an integer. 
	Note that the kernel of this map is $\frac{1}{N}\Lambda / \Lambda$ and is isomorphic to $\Z / N \Z \times  \Z / N \Z$. We denote this kernel as $E[N]$ and is the set of all $N$ torsion points in $\C / \Lambda$. This can be seen in \cref{fig:Ntorsion_subgroup}.
\end{example}
	\begin{figure}[h]
		\incfig{NTorsion_subgroup}	
		\caption{The $N$-torsion ($N = 4$) subgroup $E[N]$ of a complex torus, generated by  $\omega_1 / N$ and $\omega_2 / N$.}
		\label{fig:Ntorsion_subgroup}
	\end{figure}
\begin{example}
	Another example are the \emph{Cyclic quotient maps}. 
	Let  $\C / \Lambda$ be a complex torus and  $N \in \N$.
	If take an $N$ torsion point, $\lambda$, of $\C / \Lambda$ (a point in $\frac{1}{N} \Lambda / \Lambda$) then we can consider the subgroup $C$ generated by $\lambda$. Note that $C + \Lambda$ is a super lattice of $\Lambda$. Then the map \[
		\pi: \C \ \Lambda \to \C / (C + \lambda): z + \Lambda \to z + C + \Lambda
	\]  
	is an isogeny with kernel $C$. 
\end{example}

These two examples are very important, as they allow the construction of every isogeny. 
\begin{theorem}
	Let $\phi: \C / \Lambda \to \C / \Lambda': z + \Lambda \mapsto mz + \Lambda'$ be an isogeny. Then $\phi = \psi \circ \pi \circ [n]$, for some $\phi, \pi, [n]$, where  $\psi$ is an isomorphism, $\pi$ is a cyclic quotient map and $[n]$ is a multiply by integer map.
\end{theorem}
\begin{proof}
	Denote $K = \ker \phi$. We know that $K = m^{-1} \Lambda' / \Lambda$.
	Let $N$ be the order of $K$. So we know that every point of $K$ is $N$ torsion. 
	Hence \[
		K \subset N^{-1} \Lambda / \Lambda \simeq \Z / N \Z \oplus \Z / N \Z. 
	\]
	By the structure theorem for finite abelian groups we know that  \[
	K \cong \Z / n \Z \times \Z / n n' \Z
	\]
	for some $n, n'\in \N$. 
	This means that $nK \cong \Z / n' \Z$. Denote $\pi: \C / \Lambda \to \C / (nK + \Lambda)$ the cyclic quotient map. Now define the map \[
		\psi: \frac{\C}{nK + \Lambda} \to \frac{\C}{\Lambda'}: z + nK \mapsto  \left(\frac{m}{n}\right) z + \left( \frac{m}{n} \right)(nK + \Lambda)
	,\] 
	which is a well defined isomorphism as $mK + m\Lambda = \Lambda'$. 
	I now claim that $\phi = \psi \circ \pi \circ [n]$. Indeed, take any $z + \Lambda$ then \[
		z + \Lambda \mapsto nz + \Lambda \mapsto  nz + nK + \Lambda \mapsto  mz + \Lambda' \phi(z + \Lambda)
	.\]  
\end{proof}

\todo{maybe say something about dual maps}
\subsection{Weil Pairings} \label{sec:Weil_Pairings}
Let $E[N]$ be the  $N$ torsion subgroup of some complex torus $\C / \Lambda$, with $\Lambda = \omega_1\Z + \omega_2 \Z$. 
The \emph{Weil pairing} is a map \[
	e_N: E[N] \times E[N] \to \Z / N\Z
.\] 
Although this object comes up a lot in de discussion of complex tori/complex elliptic curves, it is defined very algebraically. 
Recall that $E[N]$ is generated by $\omega_1 / N$ and $\omega_2 / N$. So every point in $E[N]$ is  $a \cdot \omega_1 / N + b \cdot \omega_2 / N$ for some integers $a, b$ which are defined up to modulo $N$. In essense we may say that $(a, b)$ are the coordinates of point in $E[N]$. 
Now the map Weil pairing is defined as the map
\begin{align*}
	e_n: E[N] \times  E[N] &\longrightarrow \frac{\Z}{N\Z} \\
	\left(a \frac{\omega_1}{N} + b \frac{\omega_2}{N}, c \frac{\omega_1}{N} + d \frac{\omega_2}{N}\right) &\longmapsto ac-bd =  \det \begin{pmatrix} a & b \\ c& d \end{pmatrix} 
.\end{align*}
\begin{figure}[h]
	\incfig{weilpairing}
	\caption{A fun geometric interpretation of the weilpairing}
	\label{fig:weilpairing}
\end{figure}
A fun geometric interpretation of the weil pairing is given in \cref{fig:weilpairing}. As the Weil pairing is a determinant there is a geometric interpretation of it as an area. In light of Pick's theorem \cite{wiki:pick} we see that the Weil pairing is the number of points in the shaded region in the figure modulo $N$ (counting the points on the edges as half and on the vertices as $\frac{1}{4}$'th a point). 
From this interpretation it is imediately obvious that the Weil pairing is independent of the choice of basis for  $\Lambda$ as long as the orientation condition $\frac{\omega_1}{\omega_2} \in \mathcal{H}$ is satisfied. 
From the definition it is clear that $e_N$ is bilinear and skew symmetric.
As the Weil pairing is completely defined on the  $\Z / N \Z \times  \Z / N\Z$ group structure of $E[N]$ it is preserved under isomorphism.
\section{Ellpitic curves and their connection to Complex Tori}
\subsection{What are Elliptic Curves}
This section is based on section 1.2 of the book \emph{Rational Points on Elliptic Curves} by Joseph H.\ Silverman and John T.\ Tate\cite{silvermanRationalPointsElliptic2015}.
An elliptic curve is a cubic curve of the form \begin{equation}\label{eq:elliptic_curve}
y^2 = 4x^3 - g_2x - g_3
,\end{equation}
where $g_2, g_3 \in \Z$ and $g_2^3 - 27 g_3^2 \ne 0$. 
While we do not write this as an homogeneous equation we usually consider this to be its projective closure in $\mathbb{P}^2$. 
One of the most important properties of elliptic curves is that they have a certain natural abelian group structure. 
We first have to choose a zero point, $\mathcal{O} $ Usually we choose the point at infinity.
Recall B\'ezout's theorem. 
This states that any line intersects the curve \eqref{eq:elliptic_curve} in exactly 3 points, counting multiplicity.  
So for any two points $P$ and $Q$ on the curve there is exactly one line, $\ell$ that that intersects the curve in both $P$ and $Q$ (the intersection number is 2  if $P = Q$). 
Then there is a unique third intersection of this line and the elliptic curve. Let us denote this intersection by $P*Q$.
We can again take the third intersection of the the line through $P*Q$ and $\mathcal{O}$. This intersection is the result of $P + B$.
In essence we define define $P + Q =\mathcal{O} *(P*Q)$.
In the case where $\mathcal{O} $ is the point at infinity, the operation $\mathcal{O} * -$ is mirroring about the $x$ axis. So if $P * Q = (x,y)$, then $P + Q = (x, -y)$.
It can definitely be shown directly that this is a well defined abelian group structure. But we will show this in a more indirect way. 

In the next section we will bijects the points on the curve \eqref{eq:elliptic_curve} with the points of a certain complex torus and show that the group action of the elliptic curve corresponds with the addition on the complex torus.

\subsection{The Weierstrass $\wp$ Function} \label{sec:The_Weierstrass_wp_Function}
We now continue our discussion of \cite{diamondFirstCourseModular2005a}.

Elliptic curves are related to complex tori, by the meromorphic functions from the complex tori to the complex plane. These functions are also called \emph{elliptic functions}. 
A elliptic function $\phi: \C / \Lambda \to \C \cup \{\infty\} $ is naturally identified with a  $\Lambda$-periodic function $\phi': \C \to \C \cup \{\infty\}: z \mapsto \phi(z + \Lambda)$. 
We will often make use of this identification implicitly. 
One may wonder how a non-trivial example of such a function looks like. 
One example of particular interest is the \emph{Weierstrass $\wp$-function}. 
For a given lattice  $\Lambda$ we define this function to be \[
	\wp(z) = \frac{1}{z^2} + \sum_{\substack{\omega \in \Lambda \\ \omega \ne 0}}\left( \frac{1}{(z-\omega)^2} - \frac{1}{\omega^2} \right) 
.\] 
At first glance it seems obvious that such a function is $\Lambda$ periodic, but we have to be careful. 
While this sum is absolutely convergent on compact sets, its is not if we split up $\frac{1}{(z-\omega^2)} - \frac{1}{\omega^2}$ into two different summations. So we cannot simply permute the terms in the desired way. 
It still is $\Lambda$-periodic. Indeed, notice that  \[
	\wp'(z) = -2 \sum_{\omega \in \Lambda} \frac{1}{(z - \omega)^3}
\] 
this is periodic function as we can permute the terms in the sum.
We will now show that this implies that $\wp$ is $\Lambda$-periodic (exercise 1.4.2).
Note that $\wp$ is symmetric, as in this case we can permute the right terms.
For any $\omega \in \Lambda$ is holds that 
\[
	(\wp(z) - \wp(z + \omega))' = \wp'(z) - \wp'(z+ \omega) = 0 
.\] 
From this it follows that $\wp(z) - \wp(z + \omega)$ is a constant function of $z$, everywhere where it is defined.
In particular it is the case that \[
	\wp\left(-\frac{\omega}{2}\right) - \wp\left(\omega - \frac{\omega}{2}\right) = \wp\left(\frac{\omega}{2}\right) - \wp\left(\frac{\omega}{2}\right)  = 0 
.\] 
So $\wp(z) = \wp(z+ \omega)$. 
We conclude that $\wp$ is $\Lambda$-periodic.

Implicitly $\wp$ depends on the lattice $\Lambda$. 
From now on we will write  $\wp_\Lambda$ to make it clear which lattice is used in the construction of $\wp$. 
Recall from \cref{sec:modular_forms_and_complex_tori} that every lattice is isomorphic to a lattice of the form $\Lambda_\tau = \tau\Z \oplus \Z$. For these lattices we will denote $\wp_{\Lambda_\tau}$ simply as $\wp_\tau$.

Recall that \emph{Eisenstein-series} $G_k$ is the function defined as \[
	G_k(\tau) = \sum_{\substack{c, d \in \Z^2 \\ (c, d) \ne 0}} \frac{1}{(c\tau + d)^{k}}
.\] 
Note that this is equal to \[
	G_k(\tau) = \sum_{\substack{\omega \in \Lambda_\tau \\ \omega \ne 0}} \frac{1}{\omega}
.\] 
In this spirit we extend the definition of the Eisenstein series as follows. For a given lattice $\Lambda$ and an even number  $k > 2$ we define \[
	G_k(\Lambda) = \sum_{\substack{\omega \in \Lambda \\ \omega \ne 0}} \frac{1}{\omega^{k}}
.\] 
This is indeed a very natural generalisation as $G_k(\tau) = G_k(\Lambda_\tau)$.

We can describe the Laurent expansion of $\wp$ using these Eisenstein series.
\begin{proposition}\label{prop:laurent_weierstrass}
  Let $\Lambda$ be a lattice. Then the Laurent expansion of $\wp_\Lambda$ is \[
	  \wp_\Lambda(z) = \frac{1}{z^2} \sum_{\substack{n = 2 \\ n \text{ even}}}^{\infty}(n+ 1) G_{n + 2}(\Lambda)z^{n}
  .\] 
\end{proposition}
\begin{proof}
	As we are dealing with analytical functions it is sufficient to show equality on an non-empty open. We choose a neighbourhood of 0 such that in that for every $z$ in that neighbourhood $|z| < |\omega|$ for all $\omega \in \Lambda \setminus \{ 0\} $. It that case the following calculation is valid. 
\begin{align*}
	\frac{1}{(z- \omega)^2} - \frac{1}{\omega^2} &= \frac{1}{\omega^2}\left( \left( \frac{1}{(1-z / \omega)} \right) ^2 - 1 \right)  \\
						     &= \frac{1}{\omega^2}\left( \left( \sum_{n = 0}^{\infty} \left(\frac{z}{\omega}\right)^{n} \right)^2 - 1  \right) \\
						     &= \frac{1}{\omega^2}\left( \sum_{n = 0}^{\infty} (n + 1)  \left( \frac{z}{\omega} \right) ^{n} - \right)  \\
						     &= \sum_{n = 1}^{\infty} (n + 1)  \frac{z^{n}}{\omega^{n+2}} \\
\end{align*}	
Putting this result in the expression for $\wp_\Lambda$ yields
\begin{align*}
	\wp(z) &= 
	\frac{1}{z^2} + \sum_{\substack{\omega \in \Lambda \\ \omega \ne 0 }} \frac{1}{(z - \omega)^2} - \frac{1}{\omega^2} \\
	       &=  \frac{1}{z^2} + \sum_{\substack{\omega \in \Lambda \\ \omega \ne 0}} \sum _{n = 1}^{\infty} (n + 1)  \frac{z^{n}}{\omega^{n+2}}\\
	       &= \frac{1}{z^2} + \sum_{n = 1}^{\infty}\sum_{\substack{\omega \in \Lambda \\ \omega \ne 0 }} (n + 1) \frac{z^{n}}{\omega^{n+2}} \\
\end{align*}
We can exchange the order of summation as it turns out that this is an absolutely convergent series. 
When $n$ is odd we see that every term $ (n + 1) \frac{z^{n}}{\omega^{n+2}}$ cancels against $(n + 1) \frac{z^{n}}{(-\omega)^{n+2}}$. 
So we find that 
\begin{align*}
	\wp(z) &= \frac{1}{z^2} + \sum_{\substack{n = 2\\ n \text{ even}}}^{\infty}\sum_{\substack{\omega \in \Lambda \\ \omega \ne 0 }} (n + 1) \frac{z^{n}}{\omega^{n+2}} \\
	       &=\frac{1}{z^2} + \sum_{\substack{n = 2 \\ n \text{ even}}}^{\infty} (n+1) G_{n + 2}(\Lambda) z^{n}
.\end{align*}
\end{proof}
The function $\wp$ what connects complex tori to elliptic curves. 
\begin{theorem}\label{thm:complex_tori_elliptic_curves}
	Let $\Lambda = \omega_1 \oplus \omega_2$ be a lattice. Then for every $z + \Lambda$, the function  $(\wp_\Lambda(z), \wp'_\Lambda(z))$ is a point on the following cubic equation \begin{equation} \label{eq:cubic_curve_lattice}
		y^2 = 4x^3 - g_2(\Lambda)x - g_{3}(\Lambda)
	,\end{equation}
	where $g_2(\Lambda) = 60 G_4(\Lambda), g_3(\Lambda) = 140 G_6(\Lambda)$.
	The right hand side of this equation also factors in the following way \[
		4x^3 - g_2(\Lambda) x^2 - g_3(\Lambda) = 4\left(x - \wp_\Lambda\left(\frac{\omega_1}{2}\right)\right)\left(x - \wp_{\Lambda}\left(\frac{\omega_2}{2}\right)\right)\left(x - \wp\left(\frac{\omega_1 + \omega_2}{2}\right)\right)
	.\] 
\end{theorem}
\begin{proof}
	To prove \cref{eq:cubic_curve_lattice} we will first show that the equality holds for the first few terms of the Laurent expansion and then use a little trick using the periodicity of $\wp_\Lambda$ and $\wp'_\Lambda$.
	We know that (\cref{prop:laurent_weierstrass})\[
		\wp_\Lambda(z) = \frac{1}{z^2} + 3 G_4(\Lambda) z^2 + 5G_6(\Lambda) z^{4} + \mathcal{O} (z^{6})
	.\] 
	Similarly we see that \[
		\wp'_\Lambda = - \frac{2}{z^3} + 6 G_4(\Lambda)z + 20G_6(\Lambda)z^3 + \mathcal{O} (z^{5})	
	.\] 
	Using these expansions one can easily verify that \[
		(\wp'_\Lambda(z))^2 - 4(\wp(z))^3 + g_2(\Lambda)\wp_{\lambda}(z) + g_{3}(\Lambda)    = \mathcal{O} (z^{2})
	.\] 
	This expression is holomorphic and $\Lambda$-periodic. So it's bounded. We know that it must be zero as it is zero when  $z \to 0$. 
	As a result we find \cref{eq:cubic_curve_lattice}.

	We will now prove that the right-hand side factors in the desired way.
	As we've stated before $\wp$ is an even function. So $\wp'$ is an odd function. 
	If $z$ is in $E[2]$ then  $z + \Lambda = - z + \Lambda$. 
	Putting these two results together shows that $\wp'(z) = - \wp'(-z) = -\wp'(z)$.  
	As a result $\wp'(z) = 0$.
	Then we see that $\wp(z)$ must be a root of \cref{eq:cubic_curve_lattice}. 
	In particular  $\frac{\omega_1}{2}, \frac{\omega_2}{2}, \frac{\omega_1 + \omega_2}{2}$ are roots. 
	Because the derivatives $\wp'(\omega_1 / 2 ), \wp'(\omega_2 / 2), \wp'(\omega_1 / 2 + \omega_2 /2)$ are all $0$, \cref{ex:1.4.1} in the appendices shows that $\wp(\omega_1 / 2), \wp(\omega_2 / 2), \wp(\omega_1 / 2 + \omega_2 / 2)$ are distinct.
\end{proof}

Let's try to unpack the statement of \cref{thm:complex_tori_elliptic_curves}.
The first part shows that there is a map from a comples torus to an elliptic curve 
\begin{align*}
	\frac{\C}{\Lambda} &\longrightarrow \text{elliptic curve: } Z(x^3 - a x - b -y ^2) \\
	z+\Lambda &\longmapsto \left(\wp_\Lambda(z), \wp'_\Lambda(z)\right)
.\end{align*}
We know that this is an elliptic curve as it the right hand side has three distinct roots so its determinant $a^2 - 27 b^3 \ne 0  $.
We can see that this map is bijective. By \cref{ex:1.4.1}  $\wp_\Lambda$  takes every value twice. As this is an even function a certain $x \in \C$ must be taken twice, once by $z$ once by $-z$ and because $\wp'$ is an odd function $\wp'(z) \ne \wp'(-z)$ unless $z = 0$.  For every $x$-value there are exactly two $y$ values on the elliptic curve. So this is indeed a bijection. 

We will now show that this map is in fact a group isomorphism.
Let $z_1 + \Lambda, z_2 + \Lambda$ be points in $\C / \Lambda$.
These correspond to two points $P = (\wp_{\Lambda}(z_1), \wp'_\Lambda(z_1))$ and $ Q = (\wp_{\Lambda}(z_2), \wp'_\Lambda(z_2))$ in $\C^2$. 
Let $ax + by + c = 0$ be the line that intersects the elliptic curve in both these points (with intersection number 2 if  $z_1 = z_2$).
Define the function \[
	f(z) = a \wp_{\Lambda}(z) + b \wp'_\Lambda(z) + c
.\] 
Clearly $z_1, z_2$ are two zeros of this function. Obviously the third interesction of the line with the curve corresponds to the third zero of $f$. 
By \cref{ex:1.4.1} we see that this third zero, $z_3$, must be such that $z_1 + z_2 + z_3 + \Lambda = 0 + \Lambda$. 
Hence $z_3 = - z_1 - z_2$. Its corresponding point on the elliptic curve is 
$(\wp_\Lambda(z_3), \wp'_\Lambda(z_3))$.
So $P + Q = (\wp_\Lambda(z_3), - \wp'_\Lambda(z_3)) = (\wp_\Lambda(-z_3), \wp'_\Lambda(-z_3)$. 
Thus the map respects the group action. 


From now on we will often say complex elliptic curve, instead of complex torus to highlight their interpretation as elliptic curves. Further, we will often denote complex elliptic curves with a simple symbol (often $E$) instead of always writing them as the quotient  $\C / \Lambda$. In particular we write $E_\tau  = \C / \Lambda_\tau$.

I will state, but not prove, that the map from complex elliptic curves to ordinary elliptic curves is actually functorial, meaning that every morphism between complex elliptic curves corresponds to a morphism as a variety on the corresponding ordinary curves.

\section{Modular Curves and Moduli Spaces} \label{sec:modular_curves_and_moduli_spaces}
The title of this paper is modular curves, but so far have not defined these. 
The definition of these objects is very simple, but like any object in this paper, the real suprises are it's relations to other objects. 
\begin{definition}
	Let $\Gamma \subset \SL_2\Z$ be any congruence subgroup. Then the \emph{modular curve}, $Y(\Gamma)$, is defined as the set of orbits of elements in  $\mathcal{H} $ under  the action of $\Gamma$, \[
		Y(\Gamma) = \mathcal{H} / \sim = \{\Gamma \tau \;|\; \tau \in \mathcal{H} \} 
	.\] 
\end{definition}
Recall the subgroups $\Gamma(N), \Gamma_1(N), \Gamma_0(N)$ from \cref{sec:subgroups_of_the_modular_group}. 
Often we are interested in the modular curves corresponding to these congruence subgroups, so we denote \[
	Y_0(N) = Y(\Gamma_0(N)),\;\;\; Y_1(N) = Y(\Gamma_1(N)),\;\;\; Y(N) = Y(\Gamma(N))
.\]  
 
In \cref{sec:modular_forms_and_complex_tori} we have seen that there is a bijective correspondence between equivaleny classes of complex elliptic curves and points in $Y (\SL_2\Z)$. 
One may wonder whether there there is a similar result for modular curves $Y_0(N),Y_1(N)$ and $Y(N)$.
It turns out there is. 
Instead of restricting the isomorphisms between the complex elliptic curves, we will actually give the elliptic curves extra information, resulting in \emph{enhanced elliptic curves}. 

\subsection{Enhanced Elliptic Curves and Moduli Spaces} \label{sec:enhanced_elliptic_curves_and_moduli_spaces}
There are three common ways of `adding information' to elliptic curves, creating \emph{enhanced elliptic curves}. Moduli Spaces are equivalence classes of enhanced elliptic curves.  
\begin{definition} \hspace{\linewidth}
	\begin{enumerate}
		\item An \emph{enhanced elliptic curve for $\Gamma_0(N)$} is an pair $(E, C)$ where  $E$ is a complex elliptic curve and $C$ is an cyclic subgroup of order $N$.
		Two of these curves $(E_1, C_1), (E_2, C_2)$ are equivalent if and only if there is an isomorphism from $E_1$ to $E_2$ that restricts to an isomorphism from $C_1$ to $C_2$.
		The associated \emph{moduli space} is \[
			S_0(N) = \{ \text{enhanced elliptic curves for } \Gamma_0(N)\} / \sim
		.\] 
	\item An \emph{enhanced elliptic curve for $\Gamma_1(N)$} is an pair $(E, Q)$ where  $E$ is a complex elliptic curve and $Q$ is point in $E$ of order $N$.
		Two of these curves $(E_1, Q_1), (E_2, Q_2)$ are equivalent if and only if there is an isomorphism from $E_1$ to $E_2$ that maps $Q_1$ to $Q_2$.
		The associated \emph{moduli space} is \[
			S_1(N) = \{ \text{enhanced elliptic curves for } \Gamma_1(N)\} / \sim
		.\] 
	\item An \emph{enhanced elliptic curve for $\Gamma(N)$} is an tripple $(E, P, Q)$ where  $E$ is a complex elliptic curve and $P, Q$ are points in the $N$-torsion subgroup $E[N]$ such  that the weil pairing  $e_N(P, Q) = 1$.
		Two of these curves $(E_1, P_1, Q_1), (E_2, P_2, Q_2)$ are equivalent if and only if there is an isomorphism from $E_1$ to $E_2$ that maps $P_1$ to $P_2$ and $Q_1$ to $Q_2$
		The associated \emph{moduli space} is \[
			S(N) = \{ \text{enhanced elliptic curves for } \Gamma(N)\} / \sim
		.\] 
	\end{enumerate}	
\end{definition}
\begin{figure}[h]
	\centering
	\incfig{enhanced_elliptic_curves}
	\caption{Enhanced elliptic curves}
	\label{fig:enhanced_elliptic_curves}
\end{figure}
\Cref{fig:enhanced_elliptic_curves} gives a visualisation of these enhanced elliptic curves.
The following theorem gives a generalisation of the conclusion of \cref{sec:modular_forms_and_complex_tori}, showing that there are bijections in the modular curves $Y_0(N), Y_1(N), Y(N)$ and the moduli spaces $S_0(N), S_1(N), S(N)$, as well as providing more insight in what the moduli spaces look like. 
\begin{theorem}\label{thm:correspondence_modulispaces}
	\hspace{\linewidth}
	\begin{enumerate}
		\item Every enhanced elliptic curve for $\Gamma_0(N)$ is equivalent to curve of the form $(E_\tau, \left<1 / N + \Lambda_\tau \right>)$ for some $\tau \in \mathcal{H} $. 
			Two curves in this form $(E_\tau, \left<1 / N + \Lambda_\tau \right>)$ and $(E_\sigma, \left<1 / N + \Lambda_\sigma \right>)$ are equivalent if and only if $\Gamma_0\left( N \right) \tau = \Gamma_0(N) \sigma$. There is a bijective correspondence
			\begin{align*}
				S_0(N) &\longrightarrow Y_0(N) \\
				\left[\frac{\C}{\Lambda_\tau}, \left<\frac{1}{N} + \Lambda_\tau \right>\right] &\longmapsto \Gamma_0(\N)\tau
			.\end{align*}
	
		\item Every enhanced elliptic curve for $\Gamma_1(N)$ is equivalent to curve of the form $(E_\tau, 1 / N + \Lambda_\tau )$ for some $\tau \in \mathcal{H} $. 
			Two curves in this form $(E_\tau, 1 / N + \Lambda_\tau )$ and $(E_\sigma, 1 / N + \Lambda_\sigma)$ are equivalent if and only if $\Gamma_1\left( N \right) \tau = \Gamma_1(N) \sigma$. There is a bijective correspondence
			\begin{align*}
				S_1(N) &\longrightarrow Y_1(N) \\
				\left[\frac{\C}{\Lambda_\tau}, \frac{1}{N} + \Lambda_\tau \right] &\longmapsto \Gamma_1(\N)\tau
			.\end{align*}
		\item Every enhanced elliptic curve for $\Gamma(N)$ is equivalent to curve of the form $(E_\tau, \tau / N + \Lambda_\tau, 1 / N + \Lambda_\tau )$ for some $\tau \in \mathcal{H} $. 
			Two curves in this form $(E_\tau, \tau / N + \Lambda_\tau,  1 / N + \Lambda_\tau )$ and $(E_\sigma, \sigma / N + \Lambda_\sigma,1 / N + \Lambda_\sigma)$ are equivalent if and only if $\Gamma\left( N \right) \tau = \Gamma(N) \sigma$. There is a bijective correspondence
			\begin{align*}
				S(N) &\longrightarrow Y(N) \\
				\left[\frac{\C}{\Lambda_\tau}, \frac{\tau}{N} + \Lambda_\tau, \frac{1}{N} + \Lambda_\tau \right] &\longmapsto \Gamma(\N)\tau
			.\end{align*}
	\end{enumerate}
\end{theorem}
\begin{proof}
	\begin{enumerate}
		\item Let $\left( C / \Lambda, \left< c \omega_1 / N  + d \omega_2 / N  + \Lambda\right> \right) $, where $\Lambda = \omega_1 \Z \oplus \omega_2 \Z$ be any enhanced complex elliptic curve for $\Gamma_0(N)$.
			As $c \omega_1 / N + d \omega_2 /N$ generates a cyclic group of order $N$ we know that $\gcd(c, d, N) = 1$. 
			So there exists an $x \in \Z$ such that $\gcd(c, d + xN) = 1$. 
			Note that \[
				c \frac{\omega_1}{N}  + d \frac{\omega_2}{N}  + \Lambda =  c \frac{\omega_1}{N} + (d + xN) \frac{\omega_2}{N}+ \Lambda
		.\]
		We may as well have assumed from the start that $c, d$ are coprime.
		So there are $a, b \in \Z $ such that $ac - bd = 1$. 
		This means that the matrix  $\begin{pmatrix} a & b \\ c& d \end{pmatrix}  \in \SL_2\Z$ and hence invertible. 
		Let \[
		\begin{pmatrix} \omega_1' \\ \omega_2' \end{pmatrix}  = 
		\begin{pmatrix} a & b \\ c & d \end{pmatrix} 
		\begin{pmatrix} \omega_1 \\ \omega_2 \end{pmatrix} 
		.\] 
		As this is an invertible matrix it follows that $\Lambda = \omega_1' \Z \oplus \omega_2' \Z$. 
	In this new basis we see that the enhanced elliptic curve looks like $(\C / \Lambda, \left< 1 / N \omega_2'\right>)$.
	Recall from \cref{sec:modular_forms_and_complex_tori} that  $\phi: z  + \Lambda \mapsto  \frac{1}{\omega_2'} z + \Lambda_\tau$ with $\tau = \frac{\omega_1}{\omega_2}$. 
	This maps  $\left<\omega_2 / N \right>$ to $\left<1 / N + \Lambda_\tau \right>$.
	Hence  \[
		\left( \frac{\C}{\Lambda}, \left<c \frac{\omega_1}{N} + d \frac{\omega_2}{N} \right> \right) 	\simeq  
		\left( E_\tau, \left<\frac{1}{N} + \Lambda_\tau \right> \right) 
	.\] 


	
	We will now prove that \[
		\left( E_\tau , \left<\frac{1}{N} + \Lambda_\tau \right> \right) \simeq 
		\left( E_\sigma, \left<\frac{1}{N} + \Lambda_\sigma \right> \right) 
	\]
	if and only if $\tau = \gamma \cdot \sigma$ for some $\gamma \in \Gamma_0(N)$.
	Recall that $\Gamma_0(N)$ is the set of all matrices $\begin{pmatrix} a & b \\ c & d \end{pmatrix} \in \SL_2\Z$ where $c$ is divisible by $N$. 
	\ltr Suppose that $\phi: z \mapsto m z$ is such an isomorphism. Then $m \Lambda_\tau = \Lambda_\sigma$. Hence \[
	\begin{pmatrix} m \tau \\ m \end{pmatrix}  = 
	\begin{pmatrix} a & b \\ c & d \end{pmatrix} \begin{pmatrix} \sigma \\ 1 \end{pmatrix} 
	,\] 
	for some $\gamma = \begin{pmatrix} a & b \\ c & d \end{pmatrix}  \in \SL_2\Z$. 
	We see that $\tau = \gamma \cdot \sigma$. If we can show that  $N \mathbin{\mid} c$ we are done.
	We know that $m = c \sigma + d$. 
	We know that $\phi$ maps $1 / N + \Lambda_\tau$ to a generator of $\left<1 / N + \Lambda_\sigma \right>$. 
	So $\phi( 1 / N + \Lambda_\tau)$ as $0$ in it's $\sigma / N$ coordinate.
	Putting this together yields \begin{align*}
		m\left(\frac{1}{N} + \Lambda_\tau\right) &= (c \tau + d) \frac{1}{N} + \Lambda_\sigma \\ 
		&= c \frac{\sigma}{N} + d \frac{1}{N} + \Lambda_\sigma \\
	\end{align*}
	So we see that $N \mathbin{|} c$. 
	\rtl Suppose that $\tau = \gamma \cdot \sigma$ for some \[
		\gamma = \begin{pmatrix} a & b \\ c & d  \end{pmatrix} \in \Gamma_0(N)
	.\] 
	Define  $m = c \sigma + d$. Reversing the reasoning for the other implication yields that  $\phi: z\mapsto  m z$ is an isomorphism. 

	\item The first part is completely analogous to the case for $\Gamma_0.$ 
	We will now prove that \[
		\left( E_\tau , \frac{1}{N} + \Lambda_\tau \right) \simeq 
		\left( E_\sigma, \frac{1}{N} + \Lambda_\sigma \right) 
	\]
	if and only if $\tau = \gamma \cdot \sigma$ for some $\gamma \in \Gamma_1(N)$.
	Recall that $\Gamma_0(N)$ is the set of all matrices $\begin{pmatrix} a & b \\ c & d \end{pmatrix} \in \SL_2\Z$ where $c \equiv 0, a \equiv d \equiv 1 \mod N$. 

	\ltr Suppose that $\phi: z\mapsto  m z$ is such an isomorphism is such an isomorphism.
	Then $m \Lambda_\tau = \Lambda_\sigma$ for some $m$. So \[
	\begin{pmatrix} m \tau \\ m \end{pmatrix} = 
	\gamma \cdot  \begin{pmatrix} \sigma \\ 1 \end{pmatrix} , 
	\text{ for some } \gamma = \begin{pmatrix} a & b \\ c & d \end{pmatrix} \in \SL_2 \Z
	.\] 
	Then $m = c \sigma + d$. So 
	 \begin{align*}
		 m \left( \frac{1}{N} + \Lambda_\tau \right)  &= c \frac{\sigma}{N} + d \frac{1}{N} + \Lambda_\sigma \\
		 &=  \frac{1}{N} + \Lambda_\sigma
	.\end{align*}
	From this we see that $c \equiv 0, d \equiv 1 \mod N$. Recall that  $ad - bc = 1$. 
	So  $ad - bc \equiv a\cdot 1 + b\cdot 0 \equiv a \equiv 1$. So $\gamma \in \Gamma_1(N)$.
	\rtl Suppose that $\tau = \gamma \cdot \sigma$ for some \[
		\gamma = \begin{pmatrix} a & b \\ c & d  \end{pmatrix} \in \Gamma_1(N)
	.\] 	
	Define $m = c \sigma + d$. Reversing the argument for the other implication yields that $\phi:z \mapsto m z$ is an isomorphism.		

	\item Let 
		\[\left( C / \Lambda, a \frac{\omega_1}{N} + b \frac{\omega_2}{N} + \Lambda,  c \frac{\omega_1}{N}  + d \frac{\omega_2}{N}  + \Lambda \right) ,\]
	where $\Lambda = \omega_1 \Z \oplus \omega_2 \Z$ be any enhanced complex elliptic curve for $\Gamma(N)$.
	As the Weil pairing of the two points is 1 we know that $ad - bc \equiv 1 \mod N $. 
	It can be shown that we can find $a', b', c', d' \equiv a, b, c, d \mod N$ such that  $a'd' - b'c' = 1$. 
	We will assume that this has been the case from the beginning.  This means that the matrix  $\begin{pmatrix} a & b \\ c& d \end{pmatrix}  \in \SL_2\Z$ and hence invertible.  Let \[
		\begin{pmatrix} \omega_1' \\ \omega_2' \end{pmatrix}  = 
		\begin{pmatrix} a & b \\ c & d \end{pmatrix} 
		\begin{pmatrix} \omega_1 \\ \omega_2 \end{pmatrix} 
	.\] 
	As this is an invertible matrix it follows that $\Lambda = \omega_1' \Z \oplus \omega_2' \Z$. 
	In this new basis we see that the enhanced elliptic curve looks like $(\C / \Lambda, 1 / N \omega_1,  1 / N \omega_2)$.
	Recall from \cref{sec:modular_forms_and_complex_tori} that  $\phi: z  + \Lambda \mapsto  \frac{1}{\omega_2'} z + \Lambda_\tau$ with $\tau = \frac{\omega_1}{\omega_2}$. 
	This maps  $\left<\omega_2 / N  + \Lambda \right>$ to $\left<1 / N + \Lambda_\tau \right>$.
	Hence  \[
		\left( C / \Lambda, a \frac{\omega_1}{N} + b \frac{\omega_2}{N} + \Lambda,  c \frac{\omega_1}{N}  + d \frac{\omega_2}{N}  + \Lambda \right)
		\simeq
		\left( E_\tau, \frac{\tau}{N} + \Lambda_\tau, \frac{1}{N} + \Lambda_\tau \right)
	.\] 

	We will now prove that \[
		\left( E_\tau , \frac{\tau}{N} + \Lambda_\tau, \frac{1}{N} + \Lambda_\tau \right) \simeq 
		\left( E_\sigma, \frac{\sigma}{N} + \Lambda_ \sigma, \frac{1}{N} + \Lambda_\sigma \right) 
	\]
	if and only if $\tau = \gamma \cdot \sigma$ for some $\gamma \in \Gamma(N)$.
	Recall that $\Gamma(N)$ is the set of all matrices $\begin{pmatrix} a & b \\ c & d \end{pmatrix} \in \SL_2\Z$ where $c \equiv d \equiv 0, a \equiv d \equiv 1 \mod N$. 

	\ltr Suppose that $\phi: z\mapsto  m z$ is such an isomorphism is such an isomorphism.
	Then $m \Lambda_\tau = \Lambda_\sigma$ for some $m$. So \[
	\begin{pmatrix} m \tau \\ m \end{pmatrix} = 
	\gamma \cdot  \begin{pmatrix} \sigma \\ 1 \end{pmatrix} , 
	\text{ for some } \gamma = \begin{pmatrix} a & b \\ c & d \end{pmatrix} \in \SL_2 \Z
	.\] 
	Then $m = c \sigma + d$. So 
	 \begin{align*}
		 m \left( \frac{1}{N} + \Lambda_\tau \right)  &= c \frac{\sigma}{N} + d \frac{1}{N} + \Lambda_\sigma \\
		 &=  \frac{1}{N} + \Lambda_\sigma
	.\end{align*}
	From this we see that $c \equiv 0, d \equiv 1 \mod N$. 
	\begin{align*}
		m(\frac{\tau}{N} + \Lambda_\tau) &= \frac{m\tau}{N} + \Lambda_\sigma \\
		&= a \frac{\sigma}{N} + b \frac{1}{N} + \Lambda_\sigma \\
		&= \frac{\sigma}{N} \\
	.\end{align*}
	So $a \equiv 1, b \equiv 0 \mod N$
	It follows that $\gamma  \in \Gamma(N)$. 

	\rtl Suppose that $\tau = \gamma \cdot \sigma$ for some \[
		\gamma = \begin{pmatrix} a & b \\ c & d  \end{pmatrix} \in \Gamma(N)
	.\] 	
	Define $m = c \sigma + d$. Reversing the argument for the other implication yields that $\phi:z \mapsto m z$ is an isomorphism.
	\end{enumerate}
\end{proof}

\subsection{Reintepreting Modular Forms} \label{sec:reintepreting_modular_forms}

There is a bijection between the weakly modular forms of degree $k$ on $\mathcal{H}$ and the so called  \emph{degree $k$ homogenous functions with respect to $\Gamma$ }, which are functions on the set of enhanced elliptic curves for $\Gamma$. 
\begin{definition}
	Let $\Gamma$ be any of the three types of congruence subgroups we've already discussed. 
	A function \[
	F: \{\text{enhanced elliptic curves for $\Gamma$}\} \to \C
	\]
	is called \emph{homogenous with respect to $\Gamma$} if for every to enhanced elliptic curves for $\Gamma$, $E_1, E_2$ that are isomorphic with $\phi:E_1 \to E_2: z\mapsto m z $.
		\begin{align*}
			F\left(E_2\right) &= m^{-k}F\left(E_1\right) 
		.\end{align*}
\end{definition}
Given such a function $F$ we can define the function $f: \mathcal{H} \mapsto  \C$  as \[
	f(\tau) = \begin{cases}
		F(E_\tau, \left<1 / N + \Lambda_\tau\right>), \\
		F(E_\tau, 1 / N + \Lambda_\tau), \\
		F(E_\tau, \tau / N + \Lambda_\tau, 1 / N + \Lambda_\tau). \\
	\end{cases}
\] 
The case considered depends on the conrguence subgroup $\Gamma$.
It turns out that  $f$ is weight-$k$ invariant with respect to $\Gamma$. Indeed by \cref{thm:correspondence_modulispaces} if $\tau = \gamma \sigma$ for some  $\gamma = \begin{pmatrix} a & b \\ c & d \end{pmatrix}  \in \Gamma$ then the enhanced elliptic curves  $E_\tau, E_\sigma$ are isomorphic with isomorphism $\phi: E_\tau \to E_\sigma: z \mapsto (c \sigma + d) z$.
From this it follows that \[
	f(\gamma\cdot \sigma) = f(\tau) = F(E_\tau) = (c \sigma + d)^{k}F(E_\sigma) = (c \sigma + d)^{k}f(\sigma)
.\]
So $f(\tau )$ is weakly modular of degree $k$.

Conversely if $f: \mathcal{H}  \to \C$ is weakly modular of degree $k$, then we can define a function $F: E_\tau\mapsto f(\tau)$. 
\Cref{thm:correspondence_modulispaces} shows that every enhanced elliptic curve is isomorphic to some $E_\tau$ for some $\tau \in \mathcal{H} $. So we can extend $F$ uniquely to a homogeneous function on the set of enhanced elliptic curves for  $\Gamma$. 
It is an easy exercise to see that this extension is well defined. 
\section{Appendices}
\begin{theorem}
	The modular group $\SL_2(\Z)$ is generated by \[ 	
	\beta = \begin{pmatrix}  1 & 1 \\ 0 & 1 \end{pmatrix} 
	\text{ and }
	\gamma =  \begin{pmatrix} 0 & -1 \\ 1 & 0 \end{pmatrix} 
.\] 
\end{theorem}
\begin{proof}
	Let $\mathcal{A} $ be the group generated by the two matrices. 
	Suppose that $\SL_2(\Z) \setminus \mathcal{A} $ is not empty.
	Then we can choose a matrix \[
		A =\begin{pmatrix} a & b \\ c& d\end{pmatrix} \in \SL_2\Z \setminus \mathcal{A} 
	\] 
	with mimimal 1-norm, i.e. $|a| + |b| + |c| + |d|$ is minimal.
	We know that the determinant is one, hence $ad- bc = 1$. From this a couple things follow. Atleast one of  $|a|, |b|,|c|,|d|$ is greater or equal to 3 because all the matrices where $a, b, c, d = \pm 1, \pm 2$ can clearly be constructed using the generators. Suppose that without loss of generality that $|a|$ is a largest value among these (otherwise we can multiply by $\gamma$ on the left or right side to bring the largest value to the top left corner).
	As  $ab - bc = 1$ we can see that neither  $b, c, d$ can be equal to $\pm a$.  
	So  $|a|$ is not just a largest, it is the largest. 
	We can also see that $|d|$ cannot be the second largest. 
	Suppose without loss of generality that $|c|$ is the second largest value (otherwise the same argument holds with multiplication on the right).
	I claim that one of the following matrices will have smaller 1-norm than $A$ \[
		\beta A = \begin{pmatrix} a + c & b + d  \\ c & d\end{pmatrix}, \; \beta^{-1} A = \begin{pmatrix} a - c & b - d \\ c & d \end{pmatrix} 
	.\]
	Either $|a + c| = |a| - |c|$ or  $|a - c| =  |a|  - |c|$. Suppose it's the first case. 
	Then the 1-norm of $\beta A$ is
	 \[
	|a + c| + |b + d| + |b| + |d| \le |a| - |c| + |b| + 2|d| < |a| + |b| + |c| + |d|
	.\] 
	So this is a matrix with smaller 1-norm than $A$ in $\SL_2\Z$ that cannot be generated by $\beta$ and  $\gamma$. 
	This contradicts the minimality of $A$. 
\end{proof}
\begin{exercise}[exercise 1.4.1 in \cite{diamondFirstCourseModular2005a}] \label{ex:1.4.1}
	Let $\Lambda = \omega_1 \Z + \omega_2 \Z$ be a lattice,  $f$ be a non-constant elliptic function on $\C / \Lambda$ and $P = \{x \omega_1 + y\omega_2 \;|\: x, y \in [0,1]\} $ be a fundamental parallellogram of $\Lambda$.
	 \begin{enumerate}
		\item Proof that the sum of residues on $f$ is 0. 
		\item Proof that $f$ has as many poles as zeroes (counting multiplicity). 
			Conclude that $f$ takes any particular value exact  $n$ times (counting multiplicity) for some fixed $n$. Show that in particular  $\wp_\Lambda$ takes any value twice.
		\item Show that 
			\[
				\sum_{x \in \C / \Lambda} \nu_x(f) x \in \Lambda
			,\]
			where $\nu_x(f)$ is the order of $f$ at  $x$. 
	\end{enumerate}
\end{exercise}
\begin{proof}
	Let $\partial P$ be the counter clockwise oriented boundary of $P$. Because $f$ has a finite number of poles and zeros we can translate this boundary by some $t$ such that $\partial P + t$ does not contain any poles or zeroes.
\begin{figure}[h]
	\incfig{ex141}	
\end{figure}
\begin{enumerate}
	\item First note that boundary integral can be split up by integrating over the edges of $P + t$ separately\[
			\int_{\partial P+ t} f(z) \mathrm d z = \left(\int_t^{\omega_2 + t} + \int_{\omega_2+t}^{\omega_1 + \omega_2 + t} + \int_{\omega_1 + \omega_2 + t} ^{\omega_1 + t} + \int_{\omega_1}^{t}\right) f(z) \mathrm dz
	.\] 
	Note that the integrals corresponding with the two blue edges in the picture will cancel by the $\Lambda$ periodicity of $f$. Similarly the two red edges cancel. 
	Hence \[
		\int_{\partial P + t} f(z) \mathrm d z = 0
	.\] 
	By the residue theorem we find that the sum of residues on $f$ is zero.
	
\item By the same reasoning we find that \[
		\frac{1}{2\pi i}\int_{\partial P + t} \frac{f(z)}{f'(z)}dz = 0
.\] 
So by the argument principle we find that $f$ has as many poles as zeros. Let $n$ be the number of poles of $f$ and $w$ be any point in $\C$. 
Then $f - w$ also has $n$ poles and by the argument principle also $n$ zeros. 
Hence $f$ takes the value $w$ exactly $n$ times (counting multiplicity).

The Weierstrass function $\wp_{\Lambda}$ has only one pole of order 2. Hence in this case  $n = 2$.
\item 
	We will show this by first calculating \[
		\frac{1}{2\pi i}\int_{\partial P + t} z \frac{f(z)'}{f(z)} \mathrm d z
	\]
	and showing that this value must be in $\Lambda$.
	Like in the first part of this exercise we will split up the boundary in its edges.
	\[	
		\int_{\partial P+ t} z \frac{f'(z)}{f(z)} \mathrm d z = \left(\int_t^{\omega_2 + t} + \int_{\omega_2+t}^{\omega_1 + \omega_2 + t} + \int_{\omega_1 + \omega_2 + t} ^{\omega_1 + t} + \int_{\omega_1}^{t}\right) z \frac{f'(z)}{f(z)} \mathrm dz
	.\]
	By $\Lambda$-periodicity we know that the factor $f'(z) / f(z)$ is the same on opposing edges. Hence we can write \begin{align*}
		\int_{\partial P + t} z \frac{f'(z)}{f(z)} \mathrm d z &= \int_t^{\omega_1 + t} (z + \omega_2 - z) \frac{f'(z)}{f(z)} \mathrm d z + \int_{t}^{\omega_2 + t} (z - z - \omega_1) \frac{f'(z)}{f(z)} \mathrm d z\\
								       &= \omega_2 \int_t^{\omega_1 + t} 1 \; \mathrm d \ln f(z) - \omega_1 \int_t^{\omega_2 +t} 1 \; \mathrm d \ln f(z) \\
	.\end{align*} 
	From this we see that the integrals must be integer multiples of  $2\pi i$ as $f(z)$ takes the same value on the boundary points. 
	Hence 
	\[
		\frac{1}{2\pi i } \int_{\partial P + t} z \frac{f'(z)}{f(z)} \mathrm d z \in \Lambda
	.\] 
	Now note that we can write $f(z) = \sum_{x \in \{\text{poles or zero}\} }(z - x)^{\nu_x(f)}g(z)$ for some some simple function $g$. 
	So \[
		\frac{zf'(z)}{f(z)} = \sum_{x \in \{\text{poles or zero}\} }\nu_x(f) \frac{z}{ z - x} + z \frac{g'(z)}{g(z)}
	.\] 
	Note that this function has poles with residue $x \nu_x(f)$ for every $x$ which is a pole or zero of  $f$. By the residue theorem  \[
		\frac{1}{2\pi i} \int_{\partial P + t} z \frac{f'(z)}{f(z)} \mathrm d z = \sum_{x \in \C / \Lambda} \nu_x(f)  x \in \Lambda 
	.\]  
\end{enumerate}
\end{proof}
\printbibliography
\end{document}



