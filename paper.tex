\documentclass[a4paper]{article}

\usepackage[utf8]{inputenc}
\usepackage[T1]{fontenc}
\usepackage{textcomp}
%\usepackage[dutch]{babel}
\usepackage{amsmath, amssymb, amsthm}
%\usepackage{geometry}
\usepackage{tikz-cd}
\usepackage{mdframed}
\usepackage{microtype}
\usepackage{hyperref}
\usepackage{cleveref}
\usepackage{todonotes}
% figure support
\usepackage{import}
\usepackage{xifthen}
\usepackage{pdfpages}
\usepackage{transparent}
\newcommand{\incfig}[1]{%
	\def\svgwidth{\columnwidth}
	\import{./figures/}{#1.pdf_tex}
}

\pdfsuppresswarningpagegroup=1


\newtheoremstyle{theoremdd}% name of the style to be used
  {7pt}% measure of space to leave above the theorem. E.g.: 3pt
  {7pt}% measure of space to leave below the theorem. E.g.: 3pt
  {\itshape}% name of font to use in the body of the theorem
  {0pt}% measure of space to indent
  {\bfseries}% name of head font
  {}% punctuation between head and body
  { }% space after theorem head; " " = normal interword spae
  {\thmname{#1}\thmnumber{ #2}\thmnote{ (#3)}}

\newtheoremstyle{definitiondd}% name of the style to be used
  {7pt}% measure of space to leave above the theorem. E.g.: 3pt
  {7pt}% measure of space to leave below the theorem. E.g.: 3pt
  {\rmfamily}% name of font to use in the body of the theorem
  {0pt}% measure of space to indenlt
  {\bfseries}% name of head font
  {}% punctuation between head and body
  { }% space after theorem head; " " = normal interword space
  {\thmname{#1}\thmnumber{ #2}\thmnote{ (#3)}}

  \newtheoremstyle{remarkdd}% name of the style to be used
  {7pt}% measure of space to leave above the theorem. E.g.: 3pt
  {7pt}% measure of space to leave below the theorem. E.g.: 3pt
  {\rmfamily}% name of font to use in the body of the theorem
  {0pt}% measure of space to indent
  {\itshape}% name of head font
  {}% punctuation between head and body
  { }% space after theorem head; " " = normal interword space
  {\thmname{#1}\thmnumber{ #2} \thmnote{ (#3)}}


\theoremstyle{theoremdd}
\newtheorem{theorem}{Theorem}[section]
\newtheorem{proposition}[theorem]{Proposition}
\newtheorem{lemma}[theorem]{Lemma}
\newtheorem{conjecture}[theorem]{Conjecture}
\newtheorem{corollary}[theorem]{Corollary}
\newtheorem{property}[theorem]{Property}
\theoremstyle{definitiondd}
\newtheorem{definition}[theorem]{Definition}
\theoremstyle{remarkdd}
\newtheorem{example}[theorem]{Example}
\newtheorem{remark}[theorem]{Remark}

\newcommand{\N}{\mathbb{N}}
\newcommand{\Z}{\mathbb{Z}}
\newcommand{\Q}{\mathbb{Q}}
\newcommand{\C}{\mathbb{C}}
\newcommand{\R}{\mathbb{R}}
\newcommand{\ltr}{\par \noindent \framebox[1\width]{ $\implies$ } \hspace{.2cm}}
\newcommand{\rtl}{\par \noindent \framebox[1\width]{ $\impliedby$ } \hspace{.2cm} }
\newcommand{\exercise}[1]{\begin{mdframed}
	\textit{Exercise:} #1
\end{mdframed}}
\newcommand{\bigset}[2]{ \left\{ #1 \;\middle|\; #2 \right\} }


\DeclareMathOperator{\LC}{LC}
\DeclareMathOperator{\FC}{FC}
\DeclareMathOperator{\coef}{Coef}
\DeclareMathOperator{\coker}{coker}
\DeclareMathOperator{\Id}{Id}
\DeclareMathOperator{\Ann}{Ann}
\DeclareMathOperator{\im}{Im}
\DeclareMathOperator{\SL}{SL}


\author{Micha\"el Maex}
\date{\today}
\title{Modular Curves}
\begin{document}

\maketitle

\bigskip

\tableofcontents

\pagebreak

\section{Introduction}

Modular forms, elliptic curves and modular forms are three seemingly unrelated conceps that have suprising connections. bla bla blah

\section{Modular Forms}
Modular forms are function on the complex half plane $\mathcal{H} $ that are invariant in some sense under the action of a group (or subgroups of) called the modular group.

\subsection{The Modular Group and its Action}
We begin by introducing the modular group and its action on $\mathcal{H} \cup \{\infty\} $.  

\begin{definition}
	The \emph{Modular group} are the $2\times 2$ matrices with integer coefficients and determinant $1$.
	 \[
		 \SL_2(\Z) =   \bigset{\begin{pmatrix} a& b\\ c& d \end{pmatrix}}{a, b, c, d \in \Z, ac - bd = 1}
	.\] 
\end{definition}
It can be shown that the modular group is generated by \[
	\begin{pmatrix}  1 & 1 \\ 0 & 1 \end{pmatrix} 
	\text{ and }
	\begin{pmatrix} 0 & -1 \\ 1 & 0 \end{pmatrix} 
.\] 
For a proof of this we refer to the appendices. 
We now define how this group acts on the complex half plane. 
\begin{definition}
	Let  $\gamma = \begin{pmatrix} a & b \\ c & d \end{pmatrix} $ be any element of $\SL_2\Z$ and $\tau$ be any element of $\mathcal{H} \cup \{\infty\} $.
	The multiplication of these two elements is defined as 
	\[
		\gamma(\tau) = \gamma \cdot \tau = \frac{a \tau + b}{ c \tau + d}
	.\] 
\end{definition}
Note that this group action is well defined. First we have to check that this every $\tau \in \mathcal{H} \cup \{\infty\}  $ gets mapped to another element of $\tau \in \mathcal{H} \cup \{ \infty\}$. 
Note that is is sufficient to check this for the action of the generators, which induce the maps \[
\tau \mapsto  \tau+1 \text{ and } \tau \mapsto \frac{-1}{\tau}
\]
from which it is obvious.
We can also easily see that $I_2 \cdot \tau = \tau$. 
We also see that for $\gamma = \begin{pmatrix} a & b \\ c & d \end{pmatrix} $ and $\beta = \begin{pmatrix} f & g \\ h & i \end{pmatrix} $ is holds that 
\begin{align*}
	\gamma \left( \beta \cdot  \tau\right) &=  \gamma\cdot \frac{f\tau + g}{h \tau + i} \\
					       &= \frac{a\left( \frac{f \tau + g}{h\tau + i} \right) + b}{c \left( \frac{f \tau + g}{h \tau + i} \right) + d} \\	
					       &= \frac{(af + bh)\tau + ga + bi}{(cf + dh) \tau + cg + di} \\
					       &= \begin{pmatrix} af + bh & ag + bi \\ cf + dh & cg + di \end{pmatrix} \cdot \tau \\
					       &= (\gamma \beta)\cdot \tau \\
.\end{align*}

\subsection{Subgroups of the Modular Group}
Quite often it will be useful to restrict our focus to subgroups of the modular group. 
The type of subgroup we'll most interested in are the so-called \emph{congruence subgroups}.
To define these we first have to define the so called \emph{principal congruence subgroups}.  
\begin{definition}
	For any $N \in \N$ we define the \emph{principal congruence subgroup of level $N$} to be \[
		\Gamma(N) = \bigset{\begin{pmatrix} a & b \\ c & d \end{pmatrix} \in \SL_2(\Z)}{\begin{pmatrix} a & b \\ c & d \end{pmatrix}  \equiv \begin{pmatrix} 1 & 0 \\ 0 & 1 \end{pmatrix}  (\mathrm{mod}\; N)}
	.\] 
\end{definition}

Further we define two subgroups
\begin{definition}
	For every $N \in \N$ we define $\Gamma_0$ and $\Gamma_1$ to be 
	\begin{align*}
		\Gamma_0(N) &= \bigset{\begin{pmatrix} a & b \\ c & d \end{pmatrix} \in \SL_2(\Z)}{\begin{pmatrix} a & b \\ c & d \end{pmatrix}  \equiv \begin{pmatrix} * & * \\ 0 & * \end{pmatrix}  (\mathrm{mod}\; N)} \\
		\Gamma_1(N) &= \bigset{\begin{pmatrix} a & b \\ c & d \end{pmatrix} \in \SL_2(\Z)}{\begin{pmatrix} a & b \\ c & d \end{pmatrix}  \equiv \begin{pmatrix} 1 & * \\ 0 & 1 \end{pmatrix}  (\mathrm{mod}\; N)}
	.\end{align*}
\end{definition} 
Note that 
\[
	\Gamma(N) \subset \Gamma_1(N) \subset \Gamma_0(N) \subset  \SL_2\Z
.\] 
We state without proof that $[\Gamma_1(N): \Gamma(N)] = N$, $[\Gamma_0 (N): \Gamma_1(N)]= \phi(N)$, $[\SL_2\Z: \Gamma_0(N)] = N \prod_{p \mathbin | \N}(1 + 1 /p)$. 
From this it follows that $[\Gamma(N): \SL_2\Z]= N^3 \prod_{p \mathbin | N} 1 - 1 /p^2$.
We will now give the general definition of \emph{congruence subgroups}. 
 \begin{definition}
	 A subgroup $\Gamma \subset \SL_2\Z$ is a \emph{congruence subgroup} if $\Gamma(N) \subset \Gamma$ for some $N \in \N$. In this case we say that $\Gamma$ is a congruence subgroup of level $N$. 
\end{definition}
\subsection{Defining Modular Forms}\label{sec:defining_modular_forms}

Some functions on $\mathcal{H} $ are very well behaved under this action. 

\begin{definition}
	Let $k$ be an integer. 
	A meromorphic function $f: \mathcal{H}  \to \C$ is \emph{weakly modular of weight $k$ } if \begin{equation} \label{eq:weakly_modular}
		f(\gamma(\tau)) = (ct + d)^{k}f(\tau)
	,\end{equation}
	for all $\gamma = \begin{pmatrix} a & b \\ c & d \end{pmatrix}  \in \SL_2(\Z)$ and $\tau \in \mathcal{H} $. 
\end{definition}

It's clearly enough so check \eqref{eq:weakly_modular} only for the generators of $\SL_2\Z$. I.e.\begin{equation}\label{eq:sufficient_weakly_modular}
	f(\tau + 1) = f(\tau) \text{ and } f\left(-\frac{1}{\tau}\right)  = \tau^{k}f(\tau)
.\end{equation} 
From this we see that weakly modular functions are $\Z$ periodic. 

We will now introduce the weight-$k$ operator on functions $[\gamma]_k$.
 \begin{definition}
	 Let $f: \mathcal{H}  \to \C$ be a function,  $\gamma = \begin{pmatrix} a & b \\ c & d \end{pmatrix} \in \SL_2\Z$ and $k \in \Z$ then we define the function $f[\gamma]_k$ to be 
	  \begin{align*}
		  f[\gamma]_k: \mathcal{H}  &\longrightarrow \C \\
		  \tau &\longmapsto (c\tau + d)^{-k}f(\gamma(\tau))
	 .\end{align*}
\end{definition}
This allows us to characterize  \emph{weakly modular of weight  $k$} as follows.
 \begin{lemma}
	 A function $f: \mathcal{H}  \to \C$ is weakly modular of weight $k$ if and only if for every $\gamma \in \SL_2$: \[
		 f[\gamma]_k = f
	 .\] 
\end{lemma}
In light of this characterization we extend to definition of weakly modular to congruence subsets of $\SL_2\Z$. 

\begin{definition}
	Let $\Gamma$ be a congruence subset of $\SL_2\Z$. We say that a function $f: \mathcal{H}  \to \C$ is \emph{weakly modular of weight $k$ with respect to $\Gamma$} if for every  $ \gamma \in \Gamma$:
	 \[
		 f[\gamma]_k = ff[\gamma]_k 
	.\] 
\end{definition}

We are now ready to give the definition of a modular form. 
\begin{definition}\label{def:modular_form}
	Let $\Gamma$ be congruence subgroup of $\SL_2\Z$
	A function $f: \mathcal{H}  \to \C$ is a \emph{modular form of weight $ k$ with respect to $\Gamma$} if 
	\begin{enumerate}
		\item $f$ is holomorphic on $\mathcal{H} $,
		\item $f$ is weakly modular of weight $k$ with respect to $\Gamma$, 
		\item $f[\gamma]_k$ is holomorphic at $\infty$ for every $\gamma \in \SL_2\Z$. 
	\end{enumerate}
	We denote the set of modular froms of weight $k$ with respect to $\Gamma$ as $\mathcal{M} _k(\SL_2\Z)$. 
	
	If $\Gamma = \SL_2\Z$ we say simply that $f$ is a modular form of weight $k$. 
\end{definition}
The stament 'holomorhic at $\infty$' deserves some explanation. There are many equivalent ways of defining it, but for our purposes we will take this to mean that $f$ allows a fourier expansion (not too suprising as  $f$ is $\Z$ periodic)\footnote{The real picture behing holomopich at $\infty$ is that, due to $f$ being $\Z$-periodic, $H$ can be 'rolled up' on the unit disk, where $\infty$ gets mapped to the center. In this picture  $f$ being holomorphic at $\infty$ means that  $f$ allows an extension which holomorphic on the center of the disk.}, i.e.\ \[
	f(\tau) = \sum_{n=0}^{\infty} a_n e^{n 2\pi i \tau}
,\]
for some $a_n \in \C$.
Another equivalent statement is that the limit $\lim_{\im(\tau) \to \infty} f(\tau)$ exists and is finite.

Note that it is only seccesary to verify the 3rd point for a representitive of every coset $\alpha \Gamma$ as by the weak modularity  $f[\alpha]_k = f[\alpha \gamma]_k$ for every  $\gamma \in \Gamma$. 
If  $f$ is a modular form (so $\Gamma = \SL_2\Z$) then we just have to verify that $f$ is holomorphic at $\infty$.
Note that $\mathcal{M}(\Gamma)$ is a $\C$-vector space. 
It turns out that the last condition of \cref{def:modular_form} makes $\mathcal{M} (\SL_2\Z)$ of finite dimension.  
\begin{lemma}\label{lem:product_modular_forms}
	Let $f$ be a modular form of weight $k$ and $g$ a modular form of weight $l$, then 
	$fg$ is a modular form of weight  $k + g$.
\end{lemma}
\begin{proof}
	The three conditions from definition \ref{def:modular_form} must be verified. 
	\begin{enumerate}
		\item The product of holomorphic functions is holomorphic. So $ fg$ is holomorphic on $\mathcal{H} $. 
		\item Take any  $\gamma = \begin{pmatrix} a & b \\ c & d \end{pmatrix} \in \SL_2\Z$ and $\tau \in \mathcal{H} $. Then 
			\begin{align*}
				(fg)(\gamma \cdot \tau) &= f(\gamma \cdot \tau) g(\gamma \cdot \tau)   \\
							&= (c\tau + d)^{k} f(\tau) (c \tau)^{l} g(\tau) \\
							&= (c\tau + d)^{k+l} (fg)(\tau) \\
			.\end{align*}
			So $fg$ is weakly modular of weight $k+l$. 
		\item We know that $\lim_{\im(\tau) \to \infty} f(\tau)$, $\lim_{\im(\tau) \to \infty} g(\tau)$ exists and is finite. Hence $\lim_{\im(\tau) \to \infty} fg (\tau)$ exists and is finite.		
	\end{enumerate}
\end{proof}

Finally we introduce a special type of modular called a cusp. The names comes from connection to elliptic curves \todo{check that is are elliptic curves and not modular curves} that we will discuss later.
\begin{definition}\label{def:cusp}
	A \emph{cusp form} is a modular form $f$ (with respect to a congruence subgroup  $\Gamma$) such that for every  $\alpha \in \SL_2\Z$  the leading coefficient in its Fourier expansion is 0. 
	 The $\C$-vector space of cusp forms of weight $k$ is denoted $\mathcal{S} _k(\Gamma)$.
\end{definition}
Again, it is sufficient to check this for a representitive for every coset $\alpha \Gamma$.

\subsection{Examples of Modular Forms}
We may wonder whether these types of functions exist. One trivial modular form of weight $k$ is the zero function
, but the question remain whether there are non trivial modular forms of a given weight.
If $k$ is odd the answer is negative as $f(\tau) = f(-I_2 \tau) = (-1)^{k} f(\tau) = 0$. 
If we know a few modular forms of small weight, we can construct modular forms of higher weights using \cref{lem:product_modular_forms}.

However, we can construct a non trivial modular form for every even weight greater than 2.
\begin{definition}
	The \emph{Eisenstein series of weight $k$} ($k$ even and  $k >2$) is the function
	\begin{align*}
		G_k: \mathcal{H} &\longrightarrow \C \\
		\tau &\longmapsto \sum_{\substack{(c,d) \in \Z^2 \\ (c,d)\ne (0,0) }} \frac{1}{(c\tau + d)^{k}}
	.\end{align*}
\end{definition}
We of course have to check that this is indeed a modular form of weight $k$. 
Basic analysis can be used to show that the summation is absolutely convergent for every  $\tau \in \mathcal{H}$ and uniformly convergent on compact sets.
It follows that $G_k$ is holomorphic.
Recall from \cref{sec:defining_modular_forms} that it is suficient to check weak modularity for  $\gamma = \begin{pmatrix} 1 & 1 \\ 0 & 1 \end{pmatrix} $ and $\gamma = \begin{pmatrix} 0 & -1 \\ 1 & 0  \end{pmatrix} $, i.e.\ we have to show that \cref{eq:sufficient_weakly_modular} holds.
We see 
\begin{align*}
	G_k(\tau + 1) &= \sum_{\substack{(c, d) \in \Z^2 \\ (c, d) \ne (0,0)}} \frac{1}{(c (\tau + 1) + d)^{k}}\\
			 &= \sum_{\substack{(c, d) \in \Z^2 \\ (c, d) \ne (0,0)}} \frac{1}{(c \tau + (d+c))^{k}}\ \\
.\end{align*}
But this is just a permutation of the terms of the previous sequence, as $\Z^2 \to \Z^2: (c,d) \mapsto (c, c+ d)$ is a isomorphism of groups.
Hence $G_k(\tau + 1) = G_k(\tau)$.
We have to check the second equation as well.
\begin{align*}
	G_k\left(\frac{-1}{\tau}\right) &=  \sum_{\substack{(c,d) \in \Z^2 \\ (c,d) \ne 0}} \frac{1}{(-c / \tau + d)^{k}}\\
			     &= \tau^{k} \sum_{\substack{(c, d) \in \Z^2 \\ (c,d)}} \frac{1}{(d \tau - c)^{k}} \\
			     &= \tau^{k}G_k(\tau) \\
\end{align*}
The last equality is again true as we can permute the terms as the summation is absolutely convergent.

A lengthy calculation that we will omit shows that \begin{equation}\label{eq:expansion_eisenstein}
	G_k(\tau) = 2 \zeta(k) + 2 \frac{(2\pi i)^{k}}{(k-1)!}\sum_{n = 1}^{\infty} \sigma_{k-1}(n)  e^{n 2 \pi i \tau}
,\end{equation}
whenever $ k>2$ and $k$ is even, where $\zeta$ is the Riemann zeta function and $\sigma$ is the sum of divisors function, defined as
\[
	\sigma_{k}(n) = \sum_{\substack{d \mathbin | n \\ d > 0}} d^{k}
.\] 
We've yet to find an example of a cusp. We can do this by taking the right linear combination of two independent elements of $\mathcal{M} _k(\SL_2\Z)$ in a way such that the first coefficient cancels.
In this spirit we define the \emph{discriminant function}
\begin{align*}
	\Delta = (60 G_4)^3 - 27(140G_6)^2
.\end{align*}
Note that this a modular form of weight 12. 
Using \cref{eq:expansion_eisenstein} we see that the first two coefficients are $0, (2 \pi)^{12}$. 
Hence it is a non-trivial cusp of weight 12. 

\section{Lattices and Complex Elliptic Curves}
The main idea of this section is to construct tori by considering $\C$ as an additive group modulo a certain kind of embedding of $\Z^2$. This embedding of $\Z^2$ is called a lattice.  
\begin{definition}
	A \emph{lattice in $\C$} is additive subgroup $\Lambda = \omega_1 \Z \oplus \omega_2 \Z$ where $\omega_1, \omega_2 \in \C$ are linearly independent over $\R$ and $\omega_1 / \omega 2  \in \mathbb{H}$
\end{definition}
\begin{definition}
	A \emph{complex torus} is a quotient of the complex plane by a lattice, \[
		\C / \Lambda = \bigset{z + \Lambda}{z \in \C} 
	.\] 
\end{definition}
This object is called as torus because of a geometric intuition. 
If one considers the parallelogram spanned by $\vec{\omega_1}, \vec{\omega_2}$ we see that the opposing edges are `glued' together. 
So topologically this space is homeomorphic to the torus. 

\begin{figure}[h]
	\centering
	\incfig{lattice_complex_plane}
\end{figure}

The modular group tells us when two two lattices are equal. 
\begin{lemma}
	Let $\Lambda = \omega_1\Z \oplus \omega_2\Z$ and $\Lambda' = \omega_1'\Z \oplus \omega_2\Z$ be two lattices. 
	Then $\Lambda = \Lambda'$ if and only if there exists an $\begin{pmatrix} a & b \\ c& d \end{pmatrix} \in \SL_2\Z$ such that \[
	\begin{pmatrix} \omega_1' \\ \omega_2' \end{pmatrix}  = 
	\begin{pmatrix} a & b \\ c& d \end{pmatrix} 
	\begin{pmatrix} \omega_1 \\ \omega_2 \end{pmatrix} 
	.\] 
\end{lemma}
\begin{proof}[Proof. (exercise 1.3.1)]
	\ltr
	We know that $\omega_1' \in \Lambda$, there there are $a, b \in \Z$ such that $\omega_1' = a \omega_1 + b \omega_2$. Similarly there are $c, d \in \Z$ such that $\omega_2' = c \omega_1 + d \omega_2$. 
	So  \[
	\begin{pmatrix} \omega_1' \\ \omega_2' \end{pmatrix}  = 
	\begin{pmatrix} a & b \\ c& d \end{pmatrix} 
	\begin{pmatrix} \omega_1 \\ \omega_2 \end{pmatrix} 
	.\] 
	We still have to argue that the matrix is in $\SL_2\Z$, i.e. has determinant 1. 
	Analogously we can find $f, g, h, i \in \Z$ such that \[
	\begin{pmatrix} \omega_1  \\ \omega_2 \end{pmatrix}  = 
	\begin{pmatrix} f & g \\ h& i \end{pmatrix} 
	\begin{pmatrix} \omega_1' \\ \omega_2' \end{pmatrix} 
	.\]
	Hence \[
	\begin{pmatrix} \omega_1 \\ \omega_2 \end{pmatrix}  = 
	\begin{pmatrix} f & g \\ h & i \end{pmatrix} 
	\begin{pmatrix} a & b \\ c & d  \end{pmatrix} 
	\begin{pmatrix} \omega_1 \\ \omega_2 \end{pmatrix} 
	.\]  
	From this it follows that the determinants must be either $1$ or $-1$. 
	We can see that is must be $1$ as the imaginary part of $\omega_1 / \omega_2$ and $\omega_1' / \omega_2'$ must be both positive.
	\rtl
	We can use the fact that the lattices are groups. We know that the $\Lambda'$ is generated by $\omega_1'$ and $\omega_2'$.
	We further know that $\omega_1' = a \omega_1 + b \omega_2$ so $\omega_1' \in \Lambda$. 
	Similarly we see that $\omega_2' \in \Lambda$.
	So $\Lambda' \subset  \Lambda$.

	By inverting the matrix we using completely analogous reasing that $\omega_1 \in \Lambda'$ and $\omega_2 \in \Lambda'$. So $\Lambda \subset \Lambda'$.
\end{proof}

A complex torus is a Riemann surface. This a not a report about Riemann surfaces, but to not completely shove everything under the rug I will quickly explain what a Riemann Surface is.
Recall that that smooth manifolds are spaces that are locally homeomorphic to $\R^{n}$ and equipped with a smoothly compatible atlas. 
In the same spirit Riemann surfaces are objects that look locally like $\C$ and equipped with an atlas that is holomorphicly compatible.
The complex tori are Riemann surfaces. 

I will mention one theorem about Riemann surfaces without proof. 
\begin{theorem}
	Let $f$ be a holomorphic map between compact Riemann surfaces. Then $f$ is either a map to a single point or a surjection.
\end{theorem}

\subsection{Morphisms between Complex Tori}

In our study of complex tori, holomorphic maps between them are of great importance. 
In light of this we want to know how such a maps look like. It turns our there linear transformations with some restrictions. 
\begin{proposition}
	Let $\phi: \C \ \Lambda \to \C \ \Lambda'$ be a holomorphic map between complex tori. 
	Then $\phi(z + \Lambda) = mz + b + \Lambda'$ for some $m, b \in \C$ such that $m \Lambda \subset  \Lambda'$. 
	The map is invertible if and only if $m \Lambda = \Lambda'$.
\end{proposition}
\begin{proof}
	The idea is to use topology to lift the map. As $\R^2 \simeq \C$ is the universal covering space of the torus we know that we can lift $\phi$ to a holomorphic map $\tilde \phi: \C / \Lambda \to \C$. We then can make this into a homolomorphism by first projecting $\C$ onto $\C / \Lambda$. We get a holomorphic map  $\tilde \phi: \C \to \C$.
	Take any $\lambda \in \Lambda$. 
	Then the map $f_\lambda(z) = \tilde\phi(z + \lambda) - \tilde \phi(z)$. We know that $f_{\lambda}$ is continous, but we also know that is must map to an element of $\Lambda'$, which is discrete, so it must be constant. 
	We can differentiate to find that $\tilde\phi'(z+ \lambda) = \tilde\phi'(z)$ for every $\lambda \in \Lambda$, hence it is  $\Lambda$ periodic. 

	This makes $\tilde\phi'$ bounded so its constant. 
	So $\tilde \phi(z) = mz + b$ for some $m, b \in \C$. 
	Recall that $f_{\lambda}$ is is constant and equal to element of $\Lambda'$. 
	$\phi(z+\lambda) - \phi(z) = m\lambda \in \Lambda'$. 
	Hence $m \Lambda \subset \Lambda'$.

	We will now prove the second part of the theorem.
	\ltr Suppose that the inclusion  $m\Lambda \subset \Lambda'$ is proper. 
	Take a $\lambda \in \Lambda' \setminus m\Lambda$. 
	So $\lambda / m \not\in  \Lambda$. But $\phi(\lambda m) = b + \Lambda' = \phi(\lambda)$. Thus $\phi$ is not injective. So there is no inverse.
	\rtl It can be easily checked that the map $\psi: \C / \Lambda' \to \Lambda: z + \Lambda \mapsto  \frac{z-b}{m} + \Lambda$ is an inverse. 
\end{proof}
If we only consider the maps where the contsant term $b$ is $0$ then these maps are clearly group morphisms between $\C / \Lambda$ and  $\C / \Lambda'$. 
As a result, the a holomorphic group morphism between these two tori exists if and only if $m\Lambda = \Lambda'$ for some $m \in \C$. 
A nonzero holomorphic group morphism between complex tori is called an \emph{isogeny}.
\todo{say something about the connection between complex tori and holomorphisms, beginning p27 }
\begin{lemma}
	Isogenies are surjective and have a finite kernel. 
\end{lemma}
\begin{proof}
	Let $\phi: \C / \Lambda \to \C / \Lambda'$ be an isogeny. 
	From the previous discussion we know that $\phi(z+\Lambda) = mz + \Lambda'$ for some $m \in\C$ such that $m \Lambda \subset  \Lambda'$. 
	For every  $x + \lambda'$ it's clear that  $\phi( x / m + \Lambda) = x + \Lambda'$. 
	So  $\phi$ is surjective. 
	Suppose that $\phi(z + \Lambda) = 0 + \Lambda'$. Then  $m z \in \Lambda'$. So  $z \in  \Lambda'/m$. 
	We also know that  $\Lambda \subset \Lambda' / m$. 
	The inclusion map is an injective group morphism from $\Z^2$ to  $\Z^2$ which must be of finite index. 
	So there are only finitely many $z + \Lambda$ such that  $\phi(z + \Lambda) = 0$.
\end{proof}
\begin{example}
	The so called \emph{multiply by integer maps} are isogenies. These are maps of the form \[
		[N]: \C / \Lambda \to \C / \Lambda: z + \Lambda \mapsto  Nz + \Lambda
	.\] 
	Note that the of this map is $\frac{1}{N}\Lambda / \Lambda$ and is isomorphic to $\Z / N \Z \times  \Z / N \Z$. 
\end{example}
\begin{example}
	Another example are the \emph{Cyclic quotient maps}. 
	Let  $\C \ \Lambda$ be a complex torus and  $N \in \N$.
	If take an $N$ torsion point, $\lambda$, of $\C / \Lambda$ (a point in $\frac{1}{N} \Lambda / \Lambda$) then we can consider the subgroup $C$ generated by $\lambda$. Note that $C + \Lambda$ is a super lattice of $\Lambda$. Then the map \[
		\pi: \C \ \Lambda \to \C / (C + \lambda): z + \Lambda \to z + C + \Lambda
	\]  
	is an isogeny with kernel $C$. 
\end{example}

These two examples are very important, as they allow the construction of every isogeny. 
\begin{theorem}
	Let $\phi: \C / \Lambda \to \C / \Lambda': z + \Lambda \mapsto mz + \Lambda'$ be an isogeny. Then $\phi = \psi \circ \pi \circ [n]$, where  $\psi$ is an isomorphism, $\pi$ is a cyclic quotient map and $[n]$ is a multiply by integer map.
\end{theorem}
\begin{proof}
	Denote $K = \ker \phi$. We know that $K = m^{-1} \Lambda' / \Lambda$.
	Let $N$ be the order of $K$. So we know that every point of $K$ is $N$ torsion. 
	Hence \[
		K \subset N^{-1} \Lambda / \Lambda \simeq \Z / N \Z \oplus \Z / N \Z. 
	\]
	By the structure theorem for finite abelian groups we know that  \[
	K \cong \Z / n \Z \times \Z / n n' \Z
	\]
	for some $n, n'\in \N$. 
	This means that $nK \cong \Z / n' \Z$. Denote $\pi: \C / \Lambda \to \C / (nK + \Lambda)$ the cyclic quotient map. Now define the map \[
		\psi: \frac{\C}{nK + \Lambda} \to \frac{\C}{\Lambda'}: z + nK \mapsto  \left(\frac{m}{n}\right) z + \left( \frac{m}{n} \right)(nK + \lambda)
	,\] 
	which is a well defined isomorphism as $mK + m\Lambda = \Lambda'$. 
	I now claim that $\phi = \psi \circ \pi \circ [n]$. Indeed, take any $z + \Lambda$ then \[
		z + \Lambda \mapsto nz + \Lambda \mapsto  nz + nK + \Lambda \mapsto  mz + \Lambda' \phi(z + \Lambda)
	.\]  
\end{proof}
\section{Appendices}
\begin{theorem}
	The modular group $\SL_2(\Z)$ is generated by \[ 	
	\begin{pmatrix}  1 & 1 \\ 0 & 1 \end{pmatrix} 
	\text{ and }
	\begin{pmatrix} 0 & -1 \\ 1 & 0 \end{pmatrix} 
.\] 
\end{theorem}
\begin{proof}
	Let $\mathcal{A} $ be the group generated by the two matrices. 
	Suppose that $\SL_2(\Z) \setminus \mathcal{A} $ is not empty.
	Then we can choose a matrix \[
		A =\begin{pmatrix} a & b \\ c& d\end{pmatrix} \in \SL_2\Z \setminus \mathcal{A} 
	\] 
	with mimimal 1-norm, i.e. $|a| + |b| + |c| + |d|$ is minimal.
	We know that 
\end{proof}
\end{document}
